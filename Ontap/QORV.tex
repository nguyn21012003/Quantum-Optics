\documentclass{report}
\usepackage[utf8]{vietnam}
\usepackage[utf8]{inputenc}
\usepackage{anyfontsize,fontsize}
\changefontsize[13pt]{13pt}
\usepackage{commath}
\usepackage[d]{esvect}
\usepackage{parskip}
\usepackage{xcolor}
\usepackage{amssymb}
\usepackage{slashed,cancel}
\usepackage{float}
\usepackage{indentfirst}
\usepackage{pdfpages}
\usepackage{graphicx,subcaption}
\usepackage{upgreek}
\usepackage{nccmath,nicematrix}
\usepackage{mathtools}
\usepackage{amsfonts}
\usepackage{amsmath,systeme}
\usepackage[thinc]{esdiff}
\usepackage{hyperref}
\usepackage{bm,physics,upgreek}
\usepackage{fancyhdr}
%footnote
\pagestyle{fancy}
\renewcommand{\headrulewidth}{0pt}%
\fancyhf{}%
\fancyfoot[L]{Vật lý Lý thuyết}%
\fancyfoot[C]{\hspace{6.5cm} \thepage}%


\usepackage{geometry}
\geometry{
	a4paper,
	total={170mm,257mm},
	left=20mm,
	top=20mm,
}


%\newcommand{\image}[1]{
%	\begin{center}
%		\includegraphics[width=0.5\textwidth]{pic/#1}
%	\end{center}
%}
\newcommand{\image}[1]{
\begin{figure}[H]
	\centering
	\includegraphics[width=8.0cm,height=5.0cm]{pic/#1}
	\label{#1}
\end{figure}
}
\newcommand{\q}[1]{``#1''}
\newcommand{\mean}[1]{\langle{#1}\rangle}
\renewcommand{\l}{\ell}
\newcommand{\dps}{\displaystyle}

\newcommand{\f}[2]{\dfrac{#1}{#2}}
\newcommand{\at}[2]{\bigg\rvert_{#1}^{#2} }


\renewcommand{\baselinestretch}{2.0}


\title{\Huge{Ôn tập quang lượng tử}}

\hypersetup{
	colorlinks=true,
	linkcolor=black,
	filecolor=magenta,      
	urlcolor=blue,
	pdftitle={QM3},
	pdfpagemode=FullScreen,
}

\urlstyle{same}

\begin{document}
\setlength{\parindent}{20pt}
\newpage
\author{TRẦN KHÔI NGUYÊN \\ VẬT LÝ LÝ THUYẾT}
\maketitle
\tableofcontents
\setcounter{chapter}{3}
\chapter{Buổi 3}
\section{Chuyển dời bức xạ trong nguyên tử}
\subsection{Bức xạ tự phát}
\image{bucxatuphat.png}
Khi từ mức năng lượng cao chuyển dời về mức năng lượng thấp $E_{2} > E_{1}$, \q{atom} có xu hướng tự nhiên (tự phát) chuyển từ mức cao hơn (2) đến mức thấp hơn (1) $\rightarrow$ đây chính là hiện tượng phát xạ tự phát.\\
Phát xạ photon với năng lượng
\begin{gather}
	\hbar \omega = \hbar v_{0} = E_{2} - E_{1}
\end{gather}
Mỗi loại atom có phổ phát xạ tự phát đặc trưng được xác định bởi các mức năng lượng của nó theo công thức (1).
Có thể cho chuyển dời/phân rã theo cách \q{phi bức xạ} có nghĩa là phát xạ mà không cần sự có mặt của ánh sáng nhưng vẫn có cơ chế kìm hãm. Ví dụ là phân rã nguyên tử, cơ chế phonon khi chuyển dời từ năng lượng cao xuống năng lượng thấp phát xạ ra phonon(chuyển động năng thành nhiệt năng) chứ không phải photon.
\subsection{Hệ số Einstein}
Hệ số Einstein $A$: cho ta biết xác suất trên đơn vị thời gian mà electron ở mức trên sẽ chuyển xuống mức thấp hơn bằng việc phát xạ 1 photon.\\
Tốc độ phát xạ photon tỷ lệ thuật với số nguyên tử ở trạng thái kích thích và với hệ số $A$ cho quá trình chuyển dời. Phương trình \textbf{tốc độ phát xạ}(độ dốc) cho $N_{2}(t)$
\begin{gather}
	\f{dN_{2}}{dt} = -A_{21} N_{2}
\end{gather}
trong đó $N_{2}$ là tốc độ phân rã, chỉ số $21$ là ngụ ý chuyển từ mức năng lượng xuống mức năng lượng 1. Giải phương trình (2) ta được
\begin{gather}
	N_{2}(t) = N_{2}(0) \exp(-A_{21}t) \equiv N_{2}(0) \exp(-t/\tau),
\end{gather}
trong đó
\begin{gather}
	\tau = \f{1}{A_{21}}
\end{gather}
được gọi là thời gian sống(bức xạ) của trạng thái kích thích đơn vị nano tới milisecond. Áp dụng trong phương trình Bloch bán dẫn đã giới thiệu $T_{2}$ cũng có liên quan đến thời gian sống. Ý nghĩa thời gian sống: là thời gian tồn tại ở mức năng lượng	2.
\subsection{Bức xạ hấp thụ}
Phương trình tốc độ \textbf{phát xạ hấp thụ} được cho bởi
\begin{gather}
	\f{dN_{2}}{dt} = -B_{21}^{\omega} N_{1} u(\omega)
\end{gather}
Khác với bức xạ, hấp thụ không phải là tự phát. Electron không thể tự nhảy lên trạng thái kích thích trừ khi nó nhận thêm năng lượng từ nguồn photon tới. Phương trình trên gọi là phương trình chuyển dời hấp thụ trên đơn vị thời gian, trong đó $N_{1}$ gọi là số atom ở mức 1 tại thời gian $t$, $B_{12}^{\omega}$ là hệ số B Einstein cho sự chuyển dời, và $u(\omega)$ là mật độ phổ năng lượng của trường điện từ với đơn vị là J m$^{-3}$(rad/s)$^{-1}$ với xung quanh tần số $\omega = \f{E_{2} - E_{1}}{\hbar}$ mới có thể tạo chuyển dời hấp thụ(lý do là nguyên lý bất định theo thời gian $\sigma_{E} = \Delta E \Delta t$). Phổ thực chất là miền các vạch với độ phân giải cao, khi độ phân giải thấp ta thu được một vạch năng lượng. Quy ước đo vạch là bề rộng = $\f{1}{2}$ độ cao. Xung cực ngắn thì bề rộng phổ là cực rộng.
\subsection{Bức xạ kích thích}
\image{kichthich.png}
Quá trình bức xạ kích thích bởi photon(không hấp thụ) tới để tạo ra photon. Đây là một cho những cơ chế cho Laser, cảm ứng cùng pha cùng tần số(giao thoa cộng hưởng), cùng hướng, photon phát xạ có năng lượng bằng photon tới, quá trình tạo ra Laser không có quá trình hấp thụ và tự phát. Phương trình tốc độ \textbf{phát xạ kích thích} được cho bởi phương trình
\begin{gather}
	\f{dN_{2}}{dt} = - B_{21}^{\omega} N_{2} u(\omega)
\end{gather}
trong đó $N_{2}$ là số atom ở mức 2 tại $t$, $B_{21}$ là hệ số Einstein cho chuyển dời. Phát xạ kích thích là một hiệu ứng coherence quantum dynamics.
\subsection{Xác suất phát xạ/hấp thụ}
\begin{gather}
	\left( \f{dN_{2}}{dt} \right)_{sp} = - A_{21} N_{2}
\end{gather}
ta có $N_{n}$ là số nguyên tử trên đơn vị thể đang ở mức năng lượng đã cho tại thời điểm $t$, và nó còn được gọi là phân bố ở mức năng lượng bất kì. Phương trình trên là phương trình tốc độ phân rã = xác suất. Tương tự cho phi bức xạ
\begin{gather}
	\left( \f{dN_{2}}{dt} \right)_{nr} =  - N_{2} / \tau_{nr}
\end{gather}

Tốc độ phát xạ kích thích còn được biểu diễn dưới dạng
\begin{gather}
	\f{dN_{2}}{dt} = - B_{21}^{\omega} N_{2} u(\omega) = - W_{21} N_{2}
\end{gather}
trong đó $W_{21}$ là độ chuyển dời, $\left[W_{21}\right] = t^{-1}$. Nếu mức năng lượng không suy biến(một mức năng lượng có nhiều hàm sóng/trạng thái) thì $W_{12} = W_{21}$. Nếu suy biến với trọng số $g_{1}, g_{2}$ tương ứng cho mức 1 và 2 thì
\begin{gather}
	g_{1} W_{12} = g_{2} W_{21}
\end{gather}
\subsection{Bức xạ vật đen}
Xét \q{cavity} có chứa atoms, cavity hộp đen có 1 khe đủ nhỏ để khả năng ánh sáng chạy ra khỏi khe đó là cực kì nhỏ sao cho $\approx 0$ và phải không có phản xạ. Bức xạ vật đen đã có cảm ứng quá trình hấp thụ và phát xạ kích, quá trình phát xạ tự phát cũng xảy ra. \\
Khi xét thời gian đủ dài $\rightarrow$ Atom sẽ cân bằng nhiệt với bức xạ vật đen. Khi đó tốc độ/xác suất chuyển dời lên do hấp thụ phải cân bằng với xác suất chuyển dời xuống do phát xạ tự phát và phát xạ kích thích. Ở cân bằng nhiệt ta có
\begin{gather}
	B_{12} u(\omega) N_{1} = A_{12} N_{2} + B_{21} u_{\omega} N_{2}
\end{gather}
ta tìm được $u(\omega)$ để từ đó mục tiêu của chúng ta là dẫn ra các hệ số Einstein là $A$ và $B$. Atom cân bằng nhiệt với trường bức xạ ở nhiệt độ $T$ $\rightarrow$ phân bố của atom tuân theo các định luật vật lý nhiệt. Định luật Boltzmann cho tỉ số của phân bố cân bằng ở 1 và 2 tại nhiệt độ $T$
\begin{gather}
	N_{1} / N_{2} = \exp[(E_{2} -  E_{1}) / k_{B} T] = \exp(\hbar \omega / k_{B} T)
\end{gather}
từ đó ta có thể tính được
\begin{gather}
	B_{12} = B_{21} = B\\
	\f{A_{21}}{B_{21} = \hbar \omega^{3} / \pi^{2} c^{3}}
\end{gather}
\section{Tốc độ/xác suất chuyển dời bức xạ}
\subsection{Fermi's golden rule}
\begin{gather}
	\Gamma_{1\rightarrow 2} \equiv W_{1\rightarrow 2} = \f{2\pi}{\hbar} \abs{M_{12}}^{2} g(\hbar \omega) = \f{2\pi}{\hbar} \abs{\bra{2}H^{'}\ket{1}} g(\hbar \omega)
\end{gather}
trong đó $H'= e \mathbf{r} \cdot \mathcal{E}$ là thành phần nhiễu loạn, $g$ là mật độ trạng thái đã học trong LTCR, $g dE$ là số trạng thái trong đơn vị thể tích với năng lượng trong miền $E\rightarrow E + dE$.\\
Tương tác của ánh sáng với atoms qua lưỡng cực điện
\begin{gather}
	\bra{2}H'\ket{1} \equiv \boldsymbol{\mu}_{12} \cdot \mathcal{E}
\end{gather}
trong đó $\boldsymbol{\mu}_{12} = -e\bra{2} r \ket{1} \mathbf{u}$ là moment lượng cực chuyển dời, r = x,y,z và $\mathbf{u}$ là các vector đơn vị $i,j,k$,.
\section{Conclusion}
Sự khác biệt giữa tự phát và cảm ứng
\begin{itemize}
	\item \textbf{Tự phát} Nguyên tử phát xạ sóng điện tử không có liên hệ pha xác định với sóng được bức xạ bởi atom khác. Sóng có thể phát xạ theo bất kỳ hướng nào.
	\item \textbf{Cảm ứng} Được kích thích bởi sóng tới $\rightarrow$ phát xạ của 1 atom thêm vào cùng pha và cùng hướng với sóng tới. Nói cách khác, phát xạ kích thích là một hiệu ứng đồng bộ, kết hợp lượng tử trong đó photon phát xạ phải cùng pha, tần số, cùng hướng, cùng năng lượng với các photon cảm ứng/kích thích chuyển dời. Ngay cả khi bơm vào photon đi vào vẫn có bức xạ tự phát, tuy nhiên laser siêu bền nên ta bơm cùng lúc để tạo ra cường độ mạnh.
\end{itemize}

\section{Bài tập}
\subsection{Quy tắc lọc lựa}
Ma trận lưỡng cực điện được cho bởi phương trình:
\begin{align}
	M_{12} = - \boldsymbol{\mu}_{12} \cdot \boldsymbol{{ \mathcal{E} }}_0,
\end{align}
với
\begin{align}
	\boldsymbol{\mu}_{12} = -e \left( \bra{2}x\ket{1} \hat{\textbf{\textsc{i}}} + \bra{2}y\ket{1} \hat{\textbf{\textsc{j}}} + \bra{2}z\ket{1} \hat{\textbf{\textsc{k}}} \right)
\end{align}
là moment lưỡng cực điện của sự chuyển dời từ trạng thái $1\rightarrow 2$. Khi có quá nhiều trạng thái cần phải tính toán, quy tắc lọc lựa sẽ cho phép chúng ta kết luận rằng một số thành phần phần tử ma trận nhất định là bằng \textit{không} mà không cần phải tính toán một cách chính xác.

Quy tắc lọc lựa lưỡng cực điện liên quan tới các bộ số lượng tử $l,m,s$ và $m_s$, được tổng hợp trong bảng dưới.

\begin{table}[h!]
	\centering
	\begin{tabular}{c | c}
		\hline
		Số lượng tử  & Quy tắc lọc lựa                      \\ [0.6ex]
		\hline
		Tính chẵn/lẻ & Thay đổi                             \\
		$\l$         & $\Delta \l = \l' - \l = \pm 1$       \\
		$m$          & $\Delta m = m' - m = 0$ hoặc $\pm 1$ \\
		$s$          & $\Delta s = s' - s = 0 $             \\
		$m_s$        & $\Delta m_s = m_s' - m_s = 0$        \\
		\hline
	\end{tabular}
	\caption{Quy tắc lọc lựa lưỡng cực điện cho nguyên tử một electron(hydrogen-like atom).}
\end{table}
\newpage
\noindent Quy tắc:
\image{selectionrule.png}{Phân rã cho phép cho bốn mức Bohr trong nguyên tử Hydro[Griffiths, David J., and Darrell F. Schroeter. 2018. Introduction to Quantum Mechanics. 3rd ed.]}

Sự chuyển dời tuân theo quy tắc lọc lựa lưỡng cực điện được gọi là \textbf{chuyển dời cho phép}, trong khi những chuyển dời không tuân theo quy tắc thì đươc gọi là \textbf{chuyển dời cấm}.

\subsection{Dáng điệu phổ}

Xét một ``vạch phổ'', như thường được quan sát trong các môn thực nghiệm. Những tưởng ``vạch'' ấy chỉ đơn giản là một vạch theo nghĩa toán học, tức chỉ một giá trị hoàn toàn xác định. Tuy nhiên, thực tế ``vạch'' ấy lại là một tập hợp nhiều giá trị xung quanh một giá trị nào đấy, tạo nên một ``phân bố'' có dạng hình chuông, gọi là hàm ``spectral lineshape'' $g_\omega(\omega)$.
Ta dễ thấy dạng hình ``quốc dân'' của phân bố Gauss:
\image{gauss.jpg}{Phân bố Gauss}

hoặc cũng có thể là Lorentz(Hình 3).

Giá trị trung tâm được xác định bởi
\begin{align*}
	\hbar\omega_0 = E_2 - E_1.
\end{align*}
và được chuẩn hóa:
\begin{align*}
	\int_{0}^{\infty}g_\omega(\omega)d\omega = 1.
\end{align*}

\image{cauchy.png}{Phân bố Lorentz}

Đại lượng quan trọng nhất cần được xác định là Full Width at Half Maximum (FWHM), tức độ rộng tại phân nửa độ cao cực đại, cho ta biết độ rộng của phổ.\\
Thường có ba nguyên nhân ảnh hưởng đến độ rộng phổ:
\begin{itemize}
	\item Thời gian sống (lifetime (natural) broadening)
	\item Va chạm (collisional (pressure) broadening)
	\item Doppler
\end{itemize}
\subsubsection{Lifetime broadening}
Ta đã biết liên hệ giữa hệ số Einstein A và thời gian sống $\tau$. Thời gian sống này liên hệ với năng lượng bức xạ (tức sự mở rộng của phổ) qua Bất định Heisenberg $$\Delta E\Delta t \geq \hbar$$
và $$\Delta E = \hbar\Delta\omega$$
nên $$\Delta \omega=\frac{\Delta E}{\hbar} \leq \frac{1}{\tau}$$
Ta có hàm mô tả dạng phổ $$g(\omega) = \frac{\Delta\omega}{2\pi} \frac{1}{(\omega - \omega_0)^2 + (\Delta\omega/2)^2}$$
và dáng hình của nó:
\image{g(w).png}{Phân bố Lorentz. Công thức được cho bởi $g(\omega)$.}
\subsubsection{Collisional broadening}
Các phân tử có thể va đập, với nhau và với thành bình, khiến thời gian ``sống'' bị giảm đi (năng lượng bị mất nhanh hơn).\\
Bằng thống kê, ta có: $\tau_{\text{collision}} \sim \frac{1}{\sigma_s P} \left( \frac{\pi m k_B T}{8} \right)^{1/2}$ với $\sigma_s$ là tiết diện va chạm và P là áp suất. Khi ấy, collisional broadening còn được gọi là pressure broadening.\\
Trong trường hợp này, thời gian sống thường là rất ngắn, dẫn đến độ rộng phổ thật lớn so với trường hợp ``natural''.\\
Vậy nên ta có thể khắc phục bằng cách giảm thiểu áp suất P, đó là lí do ta sử dụng các đèn áp suất thấp trong khảo sát phổ.\\
\image{pbroad.png}{Cơ chế Dobler broadening.}
\subsubsection{Doppler broadening}
Việc các hạt nguyên tử di chuyển lại gần hay ra xa khỏi máy đo cũng ảnh hưởng đến phổ, khi ta xét hiệu ứng Doppler là đáng kể. \\
Khi hạt di chuyển lại gần máy đo, ta có $$\omega=\omega_0\left(1+\frac{v_x}{c}\right)$$
Số hạt di chuyển với vận tốc $v_x$ đến $v_x+dv_x$ được cho bởi cơ học thống kê: $$N(v_x) = N_0 \left( \frac{2k_B T}{\pi m} \right)^{1/2} \exp \left( -\frac{m v_x^2}{2k_B T} \right)$$ với FWHM được cho bởi $$\Delta \omega_{\text{Doppler}} = 2 \omega_0 \left( \frac{(2 \ln 2) k_B T}{mc^2} \right)^{1/2} = \frac{4 \pi}{\lambda} \left( \frac{(2 \ln 2) k_B T}{m} \right)^{1/2}$$

\subsection{Laser}
Từ ``laser'' là từ viết tắt của cụm từ ``Light Amplification by Stimulated Emission of Radiation'' (khuếch đại ánh sáng bằng phát xạ kích thích). Laser hoạt động lần đầu tiên được chứng minh vào năm 1960, và từ đó, laser đã trở thành công cụ thiết yếu trong quang học phi tuyến và lượng tử. Trong phần này, chúng ta sẽ đưa ra một cái nhìn tổng quan ngắn gọn về các nguyên lý vật lý cơ bản của hoạt động laser, và sau đó mô tả ngắn gọn các đặc tính chính của các loại laser thường được sử dụng trong phòng thí nghiệm.
\subsubsection{Phát xạ cưỡng bức (Stimulated Emission)}
Nếu như một hạt đang ở trạng thái ``cao hơn'', tạm gọi là trạng thái \textit{upper}, được chiếu bởi một nguồn sáng, hạt đó có thể chuyển dời xuống vị trí ``thấp hơn'' - \textit{lower}, xác suất chuyển dời từ \textit{upper} $\rightarrow$ \textit{lower} là như nhau đối với xác suất chuyển dời từ \textit{lower} $\rightarrow$ \textit{upper}. Quá trình này được gọi là \textbf{Phát xạ cưỡng bức} hay \textbf{Phát xạ cảm ứng}.
\image{stimulatedemission.png}{Phát xạ cưỡng bức [Griffiths, David J., and Darrell F. Schroeter. 2018. Introduction to Quantum Mechanics. 3rd ed.]}

Trường điện từ \textit{nhận} thêm năng lượng $\hbar \omega_0$ từ nguyên tử; chúng ta nói rằng một photon đi và và \textit{hai} photon bị bức ra - photon đi vào cộng với một photon từ quá trình chuyển dời được gây ra bởi photon đi vào(Hình 2). Điều này làm tăng khả năng khuếch đại, nếu như có rất nhiều nguyên tử, tất cả trong số đó đều đang ở trạng thái \textit{upper} được cảm ứng bởi một photon bay tới, một chuỗi phản ứng sẽ xảy ra. Photon bay vào tạo ra hai photon, hai photon đó tạo ra một bốn photon, và cứ thế. Chúng ta sẽ có một lượng lớn photon bay ra, tất cả chúng đều có cùng tần số. Đây chính là nguyên lý đằng sau Laser.
\subsubsection{Dao động Laser}
\image{fig4.8.png}{Sơ đồ nguyên lý của máy dao động Laser [Fox, Mark, 2006 Quantum Optics: An Introduction]}

Laser bao gồm một \textbf{buồng cộng hưởng} và hai gương đầu cuối được gọi là \textbf{gương bán mạ} và \textbf{gương phản xạ toàn phần} có độ phản xạ lần lượt là $R_1$ và $R_2$. Ánh sáng bật lại giữa hai gương đầu-cuối và được khuếch đại mỗi khi được truyền qua buồng cộng hưởng.

Khuếch đại ánh sáng xảy ra khi có buồng cộng hưởng được đo bởi hệ số khuếch đại $\gamma(\omega)$ định nghĩa bởi:
\begin{align}
	\f{d I}{d z} = \gamma(\omega) I(z),
\end{align}
với	$I$ là cường độ quang, $\omega$ là tần số góc của ánh sáng, và $z$ là hướng chuyền của chùm tia. Lấy tích phân của phương trình (3) ta được:
\begin{align}
	I(z) = I_0 e^{\gamma z}.
\end{align}
Ý nghĩa vật lý: phương trình (4) cho thấy cường độ ánh sáng tăng theo hàm exponential bên trong buồng cộng hưởng.

Xét trường hợp chùm tia sáng là gần cộng hưởng với sự chuyển dời của nguyên tử với tần số góc là $\omega_0$. Chùm tia sẽ cảm ứng với cả hấp thụ và phát xạ cưỡng bức. Để xuất hiện khuếch đại, chúng ta ràng buộc rằng tốc độ phát xạ cưỡng bức phải lớn hơn tốc độ hấp thụ để cho photon trong chùm tia được tăng lên mỗi khi được truyền qua buồng cộng hưởng:
\begin{align}
	B_{21}^{\omega} N_2 u(\omega) & > B_{12}^{\omega} N_1 u(\omega) \\
	\Rightarrow N_2               & > \f{g_2}{g_1}N_1
\end{align}
Khi xảy ra cân bằng nhiệt, tỉ lệ giữa $N_2,N_1$ được cho bởi phương trình Boltzmann
\begin{align*}
	\f{N_1}{N_2} = e^{\frac{\hbar \omega}{k_B T}},
\end{align*}
điều này có nghĩa là sẽ không bao giờ thõa mãn được phương trình (6), và cường độ ánh sáng phân rã bằng cường độ ánh sáng truyền qua bởi vì tốc độ hấp thụ	lớn hơn tốc độ phát xạ cưỡng bức. Từ đó phương trình (6) chỉ đúng khi và chỉ khi điều kiện không cân bằng. Điều này có thể quan sát được bằng cách ``bơm'' năng lượng vào buồng cộng hưởng để kích thích số lượng lớn nguyên tử đạt được trạng thái kích thích.

\subsubsection{Tính chất của Laser}
Laser được phân ra thành các loại dựa trên chất hóa học bên trong buồng cộng hưởng, đó là laser rắn, laser lỏng, laser khí. Hai đặc điểm chung cho tất cả các loại laser là tính định hướng của chùm tia và độ đơn sắc cao. Ngoài ra còn có cường độ lớn, có tính hợp cao.

\subsection{Bài tập}
\subsubsection*{Problem 4.2}
Toán tử chẵn/lẻ:
\begin{align*}
	\hat{\Pi} \psi(x) = \psi(-x).
\end{align*}
Xét ba chiều:
\begin{align*}
	\hat{\Pi} \psi(\mathbf{r}) = \psi(-\mathbf{r}).
\end{align*}
Ta biết toán tử moment lưỡng cực $\hat{\textbf{p}}_e$ tỉ lệ với $\mathbf{r}$. Khi tác động toán tử chẵn/lẻ lên $\hat{\textbf{p}}_e$ ta được:
\begin{align*}
	\hat{\Pi}^{\dagger} \hat{\textbf{p}}_e \hat{\Pi} = - \hat{\textbf{p}}_e.
\end{align*}
Ta viết lại trạng thái $\ket{2},\ket{1}$ dưới dạng bộ số lượng tử và có được:
\begin{align*}
	\boldsymbol{\mu}_{12}
	 & = \bra{n'\l'm'} \hat{\textbf{p}}_e \ket{n\l m}                                   \\
	 & =  - \bra{n'\l'm'}  \hat{\Pi}^{\dagger} \hat{\mathbf{p}}_e \hat{\Pi} \ket{n\l m} \\
	 & =  - \bra{n'\l'm'}  (-1)^{\l'} \hat{\mathbf{p}}_e (-1)^{\l} \ket{n\l m}          \\
	 & =(-1)^{\l' + \l + 1} \bra{n'\l'm'}   \hat{\mathbf{p}}_e \ket{n\l m},
\end{align*}
dấu ``$=$'' xảy ra khi và chỉ khi $ (-1)^{\l' + \l + 1} = 1 \Rightarrow \l + \l' = odd \Rightarrow \l \neq \l'$ DPCM.

\subsubsection*{Problem 4.3}
Hàm sóng cho nguyên tử Hydro được viết dưới dạng:
\begin{align}
	\psi (r,\theta,\varphi) = F(r,\theta) e^{i m_{\l} \varphi },
\end{align}
với $ m_{\l}$ là số lượng tử từ.\\
\begin{table}[h!]
	\centering
	\begin{tabular}{|l  l|}
		\hline
		$Y_{0}^{0} = \left(\frac{1}{4\pi}\right)^{1/2}$                                  & $Y_{2}^{\pm 2} = \left(\frac{15}{32\pi}\right)^{1/2} \sin^2\theta e^{\pm 2i\varphi}$                     \\
		$Y_{1}^{0} = \left(\frac{3}{4\pi}\right)^{1/2}\cos\theta$                        & $Y_{2}^{0} = \left(\frac{7}{16\pi}\right)^{1/2} (5\cos^3\theta - 3 \cos\theta)$                          \\
		$Y_{1}^{\pm1} = \mp\left(\frac{3}{8\pi}\right)^{1/2}\sin\theta e^{\pm i\varphi}$ & $Y_{3}^{\pm 2} = \mp\left(\frac{21}{64\pi}\right)^{1/2} \sin\theta (5\cos^2\theta - 1)e^{\pm 2i\varphi}$ \\
		$Y_{2}^{0} = \left(\frac{5}{16\pi}\right)^{1/2}(3\cos^2 \theta - 1)$             & $Y_{3}^{\pm3} = \left(\frac{105}{32\pi}\right)^{1/2} \sin^2\theta\cos\theta e^{\pm 3i\varphi}$           \\
		\hline
	\end{tabular}
	\caption{Một số hàm điều hòa cầu $Y_{l}^{m}(\theta,\varphi)$.}
\end{table}\\
Từ Bảng 2, ta xây dựng công thức tổng quát cho (7):
\begin{align}
	\psi_{l}^{m_{\l}} (r,\theta,\varphi) = F(r,\theta) e^{i m_{\l} \varphi }.
\end{align}
Sử dụng toán tử quay $\hat{R}_n(\phi)(n = x, y ,z)$ tác động lên (8), ta có trị riêng của $\hat{R}_z(\phi)$:
\begin{align*}
	\hat{R}_z(\phi) \psi_{l}^{m_{\l}} (r,\theta,\varphi) = \psi_{l}^{m_{\l}}(r,\theta,\varphi') = e^{-2i\pi m_\l \phi}\psi_{l}^{m_{\l}}(r,\theta,\varphi),
\end{align*}
và
\begin{align*}
	 & \hat{R}^{\dagger}_z(\phi)\hat{R}_z(\phi) = 1,                                  \\
	 & \hat{R}^{\dagger}_z(\phi) z \hat{R}_z(\phi) = z,                               \\
	 & \hat{R}^{\dagger}_z(\phi) (x\pm iy) \hat{R}_z(\phi) = e^{\mp 2i\pi} (x\pm iy).
\end{align*}
Sử dụng kí hiệu Dirac cho hàm sóng (8) và để cho phép tính được gọn gàng ta đặt $\ket{\psi_{l}^{m_{\l}} (r,\theta,\varphi)} = \ket{m_\l}$, tính các thành phần ma trận sau:
\begin{enumerate}
	\item[(a)]
	      \begin{align*}
		      \bra{m'_\l} \hat{z} \ket{m_\l}
		       & = \bra{m'_\l} \hat{R}^{\dagger}_z(\phi)\hat{R}_z(\phi) \hat{z} \hat{R}^{\dagger}_z(\phi)\hat{R}_z(\phi)\ket{m_\l} \\
		       & = e^{2i\pi m'_\l \phi} e^{-2i\pi m_\l \phi} \bra{m'_\l} \hat{z} \ket{m_\l}.
	      \end{align*}
	      Để dấu ``='' xảy ra thì $m'_\l - m_\l = 0$  (ĐPCM).
	\item[(b),(c),(d)]
	      \begin{align*}
		      \bra{m'_\l} \hat{z} \ket{m_\l}
		       & = \bra{m'_\l} \hat{R}^{\dagger}_z(\phi)\hat{R}_z(\phi) (\hat{x} \pm i\hat{y}) \hat{R}^{\dagger}_z(\phi)\hat{R}_z(\phi)\ket{m_\l} \\
		       & = e^{2i\pi m'_\l \phi} e^{-2i\pi m_\l \phi} e^{\mp 2i\pi} \bra{m'_\l} \hat{x} \pm i\hat{y} \ket{m_\l}.
	      \end{align*}
	      Để dấu ``='' xảy ra thì,  $m'_\l - m_\l \mp 1= 0$ (ĐPCM).
\end{enumerate}
\subsubsection*{Problem 4.4}
\begin{enumerate}
	\item[(a)] Ta có cường độ ánh sáng tỉ lệ với bình phương biên độ của điện trường
	      \begin{align}
		      \langle I \rangle \propto  \langle \mathcal{E}(t)^2 \rangle.
	      \end{align}
	      Từ
	      \begin{align}
		      I(t) = I(0) \exp(-t/\tau),
	      \end{align}
	      chúng ta có thể suy ra rằng
	      \begin{align}
		      \abs{\mathcal{E}(t)}^2 = \abs{\mathcal{E}_0}^2 \exp(-t/\tau).
	      \end{align}
	      Bên cạnh đó, ta có thể khai triển phương trình (11) dưới dạng hàm $\cos$
	      \begin{align}
		      \mathcal{E}(t) = \mathcal{E}_0 \cos(\omega_0 t) \exp(-t/2\tau)
	      \end{align}
	      mà vẫn thỏa mãn điều kiện rằng cường độ $I(t)$ tỉ lệ với bình phương biên độ điện trường.
	      \begin{itemize}
		      \item Với $t<0, \mathcal{E}(t) = 0$, bởi vì chùm tia sáng vẫn chưa được bức xạ ra.
		      \item Với $t \geq 0$, trường điện có công thức là (12), mô tả tính chất dao động của sóng ánh sáng với biên độ giảm dần theo hàm $\exp$, phù hợp với cường độ ánh sáng đã cho. Để tìm phổ phát xạ, chúng ta thực hiện phép biến đổi Fourier của điện trường.
	      \end{itemize}
	\item[(b)] Trường điện với $t \geq 0$ được cho bởi công thức
	      \begin{align*}
		      \mathcal{E}(t)
		       & = \mathcal{E}_0 \cos(\omega_0 t) \exp(-t/2\tau)        \\
		       & = \f{\mathcal{E}_0}{2} \cos(\omega_0 t) \exp(-t/2\tau)
	      \end{align*}
	      Phép biển đổi Fourier cho $\mathcal{E}(t)$ là
	      \begin{align}
		      \mathcal{E}(\omega) = \f{1}{\sqrt{2}} \int_{-\infty}^{+\infty} \mathcal{E}(t) e^{i\omega t} dt.
	      \end{align}
	      Khi $\mathcal{E}(t) = 0$, với $t < 0$, phương trình (13) trở thành
	      \begin{align}
		      \mathcal{E}(\omega)
		       & = \f{1}{\sqrt{2}} \int_{0}^{+\infty} \left( \f{\mathcal{E}_0}{2} e^{i(\omega_0 - \omega)t - t/2\tau} + \f{\mathcal{E}_0}{2} e^{-i(\omega_0 + \omega)t - t/2\tau} \right) dt \nonumber \\
		       & = \frac{\mathcal{E}_0}{2\sqrt{2\pi}} \left( \frac{1}{1/2\tau - i(\omega_0 - \omega)} + \frac{1}{1/2\tau + i(\omega_0 + \omega)} \right).
	      \end{align}
	      Giả sử rằng $\omega_0 \gg 1/\tau$,
	      \begin{align*}
		      \mathcal{E}(\omega) \approx \frac{\mathcal{E}_0}{2\sqrt{2\pi}} \cdot \frac{1}{1/2\tau - i(\omega_0 - \omega)}.
	      \end{align*}(ĐPCM)
	\item[(c)]  Cường độ ánh sáng tỉ lệ với bình phương biên độ, nên ta có
	      \begin{align}
		      I(\omega) \propto |\mathcal{E}(\omega)|^2 = \left|\frac{\mathcal{E}_0}{2\sqrt{2\pi}} \cdot \frac{1}{1/2\tau - i(\omega_0 - \omega)}\right|^2.
	      \end{align}
	      Ta có thể nói rằng
	      \begin{align}
		      I(\omega) \propto \f{1}{(\omega - \omega_0)^2 + (1/2\tau)^2},
	      \end{align}
	      điều này hoàn toàn đúng so với công thức được cho.
\end{enumerate}
\subsubsection*{Problem 4.6}
Ta biết khối lượng nguyên tử của $M_{Na} = 23$ và bán kính nguyên tử $r_{Na} \approxeq 0.2nm = 0.2\times 10^{-10}$, tiết diện tán xạ đàn hồi $\sigma_s$ là
\begin{align}
	\sigma_s = \pi (r_{Na})^2 = 4\pi \times 10^{-20} (m^2)
\end{align}
Ở ĐKTC, $T = 273 K$ và $P = 100k Pa$. Thời gian tán xạ đàn hồi là
\begin{align*}
	R_z
	\tau_{\text{collision}}
	 & = \f{1}{\sigma_s P} \left( \f{\pi m k_{\text{B} T}}{8} \right)^{1/2}                                                                                       \\
	 & = \f{1}{4\pi \times 10^{-20 } \times 100\times10^{3}} \left( \f{\pi \frac{23}{N_A} \times 10^{-3} \times 1.38\times 10^{-23} \times 273 }{8} \right)^{1/2} \\
	 & \approx 5.9 \times 10^{-10} (s)
\end{align*}
\chapter{Buổi 4 Thống kê Photon}
\section{Thống kê photon đếm}
Giả sử chùm ánh sáng hoàn toàn đồng bộ(coherent), đơn sắc(monochromatic), với tần số $\omega$ và cường độ(intensity) không đổi $I$.\\
Thông lượng photon(photon flux) $\Phi \equiv$ số photon trung bình đi qua tiết diện ($A$) của chùm tia trong một đơn vị thời gian $\rightarrow$ $\Phi$ bằng năng thông(energy flux)(năng lượng đi qua một diện tích trong đơn vị thời gian) chia cho năng lượng của từng photon
\begin{gather}
	\Phi = \f{IA}{\hbar \omega} = \f{P}{\hbar \omega} \text{[photon s}^{-1}]
\end{gather}
Hiệu suất lượng tử $\eta$ là tỷ lệ giữa số photon đếm được với số photon tới
$\Rightarrow$ số lượng đếm trung bình $N$ được máy dò ghi nhận trong thời gian đếm $T$:
\begin{gather}
	N(T) = \eta \Phi T = \eta \f{PT}{\hbar \omega}
\end{gather}
Tốc độ đếm(count rate) trung bình $R$
\begin{gather}
	R = N/T = \eta \f{P}{\hbar \omega} [ \text{count s}^{-1} ]
\end{gather}
Dectector cần khoảng thời gian \q{phục hồi} sau mỗi sự kiện đếm $\rightarrow$ thực tế có thời gian chết(dead time) $\approx$ 1 $\mu$s giữa các lần đếm liên tiếp $\rightarrow$ $R$ có giới hạn trên $10^{6}$ count s$^{-1}$. $\eta$ của dectector khoảng 10\%. Bộ đếm photon chỉ phù hợp cho \textbf{chùm ánh sáng yếu} $\approx 10^{-12}$ W. Chùm ánh sáng mạnh hơn cần có cách dò khác. Ta cần chú ý rằng khi đếm photon thì số photon phải là số nguyên trong mỗi đoạn chia.
\subsection{Coherent light: thống kê Poissonian}
Xét trường ánh sáng ``coherent'' có biên độ $E_{0}$ với tần số $\omega$ và pha $\phi$ với trường điện là hàm
\begin{gather}
	E(x,t) = E_{0} \sin(kx - \omega t + \phi)
\end{gather}
Cường độ $I$ $\propto$ bình phương biên độ và là hằng số nếu $E_0$ và $\phi$ không có phụ thuộc thời gian(coherent hoàn toàn) $\rightarrow$ Do đó sẽ không có thăng giáng cường độ và thông lượng photon là không đổi theo thời gian. Trong thời gian cực ngắn vẫn luôn có thăng giáng thống kê điều này xảy ra là do bản chất \textbf{rời rạc} của photon. \\
\textbf{Perfectly coherent light} với $I$ không đổi có thống kê Poissonian.\\
Xét chùm tia công suất không đổi. Số phonon trung bình trong đoạn dài $L$(tức là trong khoảng thời gian $T = L / c$) của chùm tia này được cho bởi
\begin{gather}
	\overline{n} = \Phi \f{L}{c}
\end{gather}
Giả sử $L$ là đủ lớn, để $\overline{n}$ phải có giá trị nguyên xác định rõ ràng(số photon phải là số nguyên) Chia đoạn $L$ thành $N$ đoạn nhỏ có độ dài $L/N$. Xem $N$ là đủ lớn để xác suất tìm được 1 photon trong đoạn $L/N$ bất kỳ( $p = \overline{n} / N$) là rất nhỏ(Xác suất tìm được 2 hoặc nhiều hơn photon có thể bỏ qua). Ta có
\begin{gather}
	\overline{n} = \sum_{n = 1}^{N} p(n) N(n) = p \sum_{n = 1}^{N} 1  \Rightarrow p = \overline{n} / N
\end{gather}
Ta tính được xác suất $P(n)$ này tương đương với việc tìm thấy $n$ đonạ bất kì có chứa 1 photon với $N - n$ đoạn trống. Khi có phân bố đều, xác suất này được cho bởi nhị thức Newton
\begin{equation}
	\begin{aligned}
		p
		 & = \f{\overline{n}}{N}     
		 = \f{N!}{n!(N-n)!} p^{n} (1 - p)^{N-n}
	\end{aligned}
\end{equation}
khi $N\rightarrow \infty$ thì
\begin{gather}
	P(n) = \f{\overline{n}^{n}}{n!} e^{- \overline{n}}
\end{gather}
đây chính là phân bố Poissonian. Phương sai, độ lệch chuẩn được tính như sau
\begin{gather}
	Var(n) \equiv (\Delta n)^{2} = \sum_{n} (n - \overline{n})^{2} P(n) = \overline{n} \Rightarrow \Delta n = \sqrt{\overline{n}}
\end{gather}
Có tất cả 3 loại thống kê photon, trong đó ánh sáng có perfect coherent với cường độ không đổi là ánh sáng ổn định nhất
\begin{itemize}
	\item \textbf{Sub-Poisson}: $\Delta n < \sqrt{\overline{n}}$
	\item \textbf{Poisson}: $\Delta n = \sqrt{\overline{n}}$
	\item \textbf{Super-Poisson}: $\Delta n > \sqrt{\overline{n}}$
\end{itemize}
Bằng một cách định tính: Ta có thể thấy ngay, so với ánh có cường độ không đổi(có dạng bền và ổn định nhất), ánh sáng với cường độ thăng giáng cổ điển thì thăng giáng số photon, điều này có nghĩa là khi cường độ hoàn toàn ổn định thì ánh sáng cho ra thống kê Poisson, vậy nên tất cả các chùm ánh sáng cổ điển tức là ánh sáng có cường độ thay đổi theo thời gian sẽ có phân bố Super-Poisson. Các loại ánh sáng này sẽ là \q{nhiễu} hơn ánh sáng hoàn toàn kết hợp, theo nghĩa cổ điển là chúng có các biến đổi lớn hơn về cường độ, và theo nghĩa lượng tử là chúng có các biến động số photon lớn hơn.\\
Ngược lại, ánh sáng có thống kê Sub-Poisson có phân bố hẹp hơn trường hợp Super-Poisson và do đó \q{yên tĩnh} hơn ánh sáng perfectly coherent
\subsection{Thống kê Super-Poisson}
Super-Poisson được xác định bởi:
\begin{gather}
	\Delta n > \sqrt{\overline{n}}
\end{gather}
\subsection{Thermal light(Ánh sáng nhiệt)}
Là bức xạ điện từ phát ra từ vật thể được truyền nhiệt(bức xạ vật đen). Bức xạ này có dao động
\begin{gather}
	\rho(\omega, T) d\omega = \f{\hbar \omega}{\pi^{2} c^{3}} \f{1}{\exp(\hbar \omega / k_{B} T) - 1} d\omega
\end{gather}
Có thể xem mỗi mốt là một dạng động điều hoà(giống với lý thuyết của Einstein) với tần số góc $\omega$ và năng lượng đã được lượng tử hoá
\begin{gather}
	E_{n} = (n + \frac{1}{2} \hbar \omega)
\end{gather}
Xác suất tìm thấy photon ở tần số $\omega$(nhưng chưa chắc đã cùng pha, biên độ như là ánh sáng perfectly coherent) được cho bởi định lý Boltzmann
\begin{equation}
	\begin{aligned}
		P_{\omega}(n)
		 & = \f{\exp(-E_{n} / k_{B} T)}{\sum_{n = 0}^{\infty} \exp(-E_{n} / k_{B} T)}      \\
		 & = \f{\exp(-n\hbar \omega)}{\sum_{n = 0}^{\infty} \exp(-n\hbar\omega / k_{B} T)} \\
		 & = \f{x^{n}}{\sum_{n = 0}^{\infty} x^{n}}                                        \\
		 & = x^{n}(1-x)
	\end{aligned}
\end{equation}
từ đó, số photon trung bình là
\begin{equation}
	\begin{aligned}
		\overline{n}
		 & = \sum_{n = 0}^{\infty} n P_{\omega}(n)                                                                              = \sum_{n = 0}^{\infty} n x^{n} (1-x)                                                                                  \\
		 & = (1 - x) x \f{d}{dx} \left( \sum_{n = 0}^{\infty} x^{n} \right) (\text{Tại vì đạo hàm thì số bậc của $x$ là $n - 1$}) \\
		 & = (1 - x) x \f{d}{dx} \left(\f{1}{1 - x}\right)                                                                      
		 = (1 - x) x \f{1}{(1 - x)^2}                                                                                           \\
		 & = \f{x}{1 - x}                                                                                                         = \f{\exp(-\f{\hbar \omega}{k_B T})}{1 - \exp(-\f{\hbar \omega}{k_B T})} = \f{1}{\exp(\f{\hbar \omega}{k_B T}) - 1} .
	\end{aligned}
\end{equation}
Như vậy, số photon trung bình được tính theo công thức Plank
\begin{gather}
	\overline{n} = \f{1}{\exp(\hbar \omega/k_{B}T) - 1}
\end{gather}
Thế ngược vào hàm phân bố $P_{\omega}(n)$, ta thấy
\begin{gather}
	P_{\omega}(n) = \f{1}{\overline{n} + 1} \left(\f{\overline{n}}{\overline{n} + 1}\right)^{n}
\end{gather}
tuân theo phân bố Bose-Einstein, đạt cực đại khi $n =0$ và giảm theo hàm $\exp$.
\image{BosEin.png}
Đồ thị trên so sánh phân bố Bose-Einstein của thermal light và phân bố Poisson của ánh sáng perfectly coherent, vì tính chất tự nhiên của ánh sáng nhiệt ta đang xét có năng lương không ổn định mà dao động, \q{nhiễu}. Phân bố B-E cho phương sai $\Delta n$ là luông rộng hơn phân bố Poisson
\begin{gather}
	(\Delta n)^{2} = \overline{n} + \overline{n}^{2} 
\end{gather} 
rõ ràng là $\Delta n_{\text{B-E}} > \Delta n_{\text{Poisson}}$, điều này hoàn toàn thoả mãn định nghĩa về ánh sáng có \q{Super-Poisson}. Nhắc lại rằng, tuy ta đang xem xét ánh sáng nhiệt, nhưng chỉ xét ở một mode $\omega$, nhưng thực tế, bức xạ vật đen là một chuỗi liên tục nhiều mode dao động với những tần số khác nhau. Khi đó phương sai được cho bời
\begin{gather}
	\Delta n_{\text{B-E}} = \overline{n} + \f{\overline{n}^{2}}{N_{m}}
\end{gather}
Trong thực tế, khi làm việc với thermal light, nhìn chung, vì rất khó để đo một tần số xác định(có nghĩa là $N_{m} \gg$), nên thống kê nhìn chung là phân bố Poisson. Ngoài ra theo Einstein, ta có
\begin{equation}
	\begin{aligned}
		\mean{\Delta E^{2}} d\omega 
		&= k_{B} T \f{\partial}{\partial T} (V \rho d\omega)\\	
		&= k_{B} T^{2} V d\omega \f{\partial \rho}{\partial T} \\	
		&= \left( \rho \hbar \omega + \f{ \pi^{2} c^{3} }{\omega^{2} \rho^{2} } \right) V d\omega \\
	\end{aligned}	
\end{equation}
trong đó, số hạng thứ nhất là lượng tử hoá năng lượng của trường điện từ, nói cách khác là bản chất photon(hạt) của ánh sáng, trong khi đó số hạng thứ hai ra đời tự sự thăng giáng trong năng lượng của trường điện từ, có bản chất là cổ điển, còn được gọi là \q{wave noise}.
\subsection{Chaotic light(Ánh sáng hỗn loạn)}
Chaotic light nhìn chung cũng giống với ánh sáng coherent light, nhưng có cường độ(intensity) $I$ là không ổn định mà thăng giáng theo từng khoảng thời gian[thời gian này gọi là thời gian đồng bộ hay thời gian kết hợp(coherent time)] $\tau_{c} \approx \Delta \omega^{-1}$. Có thể hiểu giống như nguồn phát có công suất không ổn định.\\
Khi này, tốc độ đếm trong thời gian $T$ được xác định bằng cách lấy tích phân thông lượng trong khoảng thời gian $t \rightarrow t + T$
\begin{gather}
	W(T) = \int_{t}^{t + T} \eta \Phi(t') dt'
\end{gather}
và \textbf{phương sai của chaotic light} được tính là
\begin{gather}
	(\Delta n)_{\text{chaotic light}}^{2} = \mean{W(T)} + \mean{\Delta W(T)^{2}}
\end{gather}
Số photon trung bình $\overline{n} = \Phi \f{L}{c}$ trong đoạn dài $L$(tức là trong khoảng thời gian$T = L/c$) chính là $\mean{W(T)}$. 
Vì $\Phi = \Phi(t)$, tức sự thay đổi trong cường độ ánh sáng trong một khoảng thời gian(nhất là khi nhỏ hơn $\tau_{c}$) là lớn, nên số hạng thứ hai là không đáng kể và có thể xem xét bỏ qua. Khi đó, phân bố sẽ trở thành phân bố Super-Poisson. Tuy nhiên, khi ta xét trên thời gian rất lớn, sự dao động theo $\tau_{c}$ bây giờ sẽ không còn đáng kể, và phân bố sẽ trở về phân bố Poisson.
\subsection{Sub-Poissonian light}
Sub-Poisson được xác định bởi
\begin{gather}
	\Delta n < \sqrt{\overline{n}}
\end{gather}
Từ đồ thị ta thấy rằng, phân bố Sub-Poisson có phân bố còn hẹp hơn cả phân bố Poisson của ánh sáng perfectly coherent. Nói cách khác, thứ ánh sáng này có cường độ ổn định hơn cả coherent light. Nhưng trong thực tế, ánh sáng Sub-Poisson không có đối tượng tương đương trong cổ điển, tức sự khảo sát phân bố này hoàn toàn thuộc về bản chất lượng tử cuả ánh sáng, nhưng không khó để ta tưởng tượng ra hệ tạo nên thứ ánh sáng có cường độ và \q{more coherent} hơn cả ánh sáng có perfectly coherent.\\
\image{tubephoton.png}
\noindent Xét một ống chứa photon (a) sao cho khoảng cách giữa các photon là $c\Delta t$ giống hệt nhau. Kết quả đo số photon trong mỗi lần do sẽ là là một số được xác định như sau
\begin{gather}
	N = \text{Int}\left(\eta \f{T}{\Delta t}\right)
\end{gather}
và thống kê sẽ có dạng như hình (b). Khi đó, $\overline{n} = N$ và ánh sáng khi này là \q{highly Sub-Poisson} với $\Delta n = 0$. Các dòng photon như hình (a) với $\Delta n =0$ được gọi là các \q{trạng thái photon}(photon number states). Nhưng trong thực tế, tạo ra các nguồn phát như vậy là không khó trong phòng thí nghiệm, nhưng để đo được chúng là một câu chuyện khác, bởi vì chúng có những \textbf{sự mất mát quan học} gây ra bởi sự suy giảm thống kê photon. Có các loại ánh sáng Sub-Poisson khác trong đó $\Delta t$ giữa các photon trong chùm tia không hoàn toàn giống nhau, nhưng các $\Delta t$ này cần đều hơn các $\Delta t$ ngẫu nhiên của loại ánh sáng Sub-Poisson. \\
Như vậy khi nhắc tới ánh sáng Sub-Poisson, thì ta cần quan tâm 2 điều đó là tìm cách nâng cao hiệu suất lượng tử $\eta$, và tạo ra được dòng photon như hình (a)
\section{Degradation of photon statistics by losses}
\image{splitBeamer.png}
Ta xét ánh sáng đi qua \q{môi trường thất thoát}(lossy medium), rồi đến đầu dò. Nếu độ truyền qua của môi trường là $T$, thì chúng ta có thể mô hình hoá sự thất thoát như một \q{bộ tách chùm tia sáng} với tỉ lệ chia là $\frac{T}{1-T}$ như hình (b). Quá trình chia chùm xảy ra ngẫu nhiên ở mức độ photon đơn lẻ, với xác suất là $T$ và $1-T$. Có thể nói rằng môi trường thất thoát này chọn ngẫu nhiên photon từ chùm tia tới với xác suất $T$. Mô hình này tiện lợi trong việc khảo sát các yếu tố khác ảnh hưởng đến hiệu suất đếm của đầu dò, trong đó gồm
\begin{itemize}
	\item Thu thập không hiệu quả, khi chỉ có một phần nhỏ ánh sáng từ nguồn được thu thập, quá trình này giống như việc \q{chọn lọc} ngẫu nhiên photon từ nguồn phát.
	\item Thất thoát do hấp thụ, tán xạ hay phản xạ khỏi bề mặt. Quá trình này đã loại bớt photon khỏi chùm
	\item Sự kiếm hiệu quả của quá trình thu thập khi sử dụng những đầu dò có hiệu suất lượng tử không cao. Quá trình này chọn lọc ngẫu nhiên photon nào sẽ được đo.
\end{itemize}
Trong đó, hai quá trình đầu làm suy giảm thống kê photon \q{tự thân}, còn quá trình cuối làm suy giảm mối liên hệ giữa thống kê photon và photonelectron. Lập luận trên cho ta thấy, ánh sáng Sub-Poisson rất nhạy cảm với những thất, và dẫn đến dễ dàng bị đưa trở về thống kê Poisson.
\section{Theory of photon dectection}
Ta khảo sát mối liên hệ giữa thống kê photon được ghi nhận bởi máy dò và thống kê photon thực của chùm tia.
\subsection{Bán cổ điển}
Cách tiếp cận bán cổ điển đó là ta xem chùm tia tới là sóng điện từ, những nguyên tử được kích thích(bức xạ) thì được xem như là lượng tử hoá. Tính chất thống kê của thời gian giữa các xung đầu ra có thể được giải thích bằng cách đưa ra ba giả định sau đây về quá trình quang điện
\begin{itemize}
	\item Xác suất phát xạ một quang electron trong một khoảng thời gian ngắn $\Delta t$ tỷ lệ thuận với cường độ $I$, diện tích $A$ được chiếu sáng và khoảng thời gian là $\Delta t$
	\item Nếu $\Delta t$ là đủ nhỏ, xác suất phát xạ hai quang electron là không đáng kể.
	\item Các sự kiện phát xạ quang điện được ghi nhận trong các khoảng thời gian khác nhau là độc lập thống kê với nhau
\end{itemize}
Ta có thể viết xác suất phát xạ một photon trong khoảng thời gian từ $t \rightarrow t + \Delta t$
\begin{gather}
	P(1;t,t + \Delta t) = \eta I(t) \Delta t
\end{gather}
với $\eta$ tỉ lệ với diện tích được chiếu sáng và bằng xác suất phát xạ trên một đơn vị thời gian trên một đơn vị cường độ. Nếu cường độ $I$ là hằng số, thì số photon đếm trung bình $\overline{n}$ trong khoảng thời gian $0 \rightarrow t$ được cho bởi 
\begin{gather}
	\overline{n} = \eta I t \equiv Ct
\end{gather}
và xác suất có dạng là
\begin{gather}
	P_{n} (t) = \f{\overline{n}^{n}}{n!} e^{- \overline{n}}
\end{gather}
cho ta phân bố Poisson.\\
Phương trình (5.55 M.Fox) chứng minh rằng chúng ta có thể giải thích thống kê đếm photon Poisson khi phát hiện ánh sáng có cường độ không đổi theo thời gian mà không cần đếm khái niệm photon. Tất cả những gì chúng ta cần là việc phát xạ quang electron là một quá trình xác suất được kích hoạt bởi viêc lượng tử năng lượng từ chùm ánh sáng. Do đó, việc phân tích thống kê đếm photon không nhất thiết cho chúng ta biết bất cứ điều gì về thống kê photon cơ bản. Đồng thời ,rõ ràng thống kê Poisson là không thể trong lý thuyết bán cổ điển. Điều này là do chúng ta thu được công thức Poisson nếu cường độ là hằng số, và nếu cường độ thay đổi theo thời gian, có thể chứng minh rằng chúng ta thu được kết quả Super-Poisson như trong phương trình (5.38 M.Fox) Do đó việc quan sát thống kê đếm photon dưới phân bố Poisson là một minh chứng rõ ràng cho thấy cách tiếp cận bán cổ điển là không đầy đủ. Như vậy những thí nghiệm Sub-Poisson chỉ có thể được/hy vọng giải thích bằng cách xử lý lượng tử đầy đủ.
\subsection{Lượng tử}
Xét thống kê đếm photon được đo trong khoảng thời gian $T$. Ta có mối liên hệ gữa phương sai của số photon đếm được $(\Delta N)^{2}$ và phương sai của số photon đến máy dò trong cùng khoảng thời gian
\begin{gather}
	(\Delta N)^{2} = \eta^{2} (\Delta n)^{2} + \eta(1-\eta) \overline{n}
\end{gather}
với $\eta = N / \overline{n}$ là hiệu suất lượng tử.
Ta thấy
\begin{itemize}
	\item Nếu $\eta = 1$, ta có $\Delta N = \Delta n$ và thăng giáng photon đếm tái tạo đúng thăng giáng của dòng photon tới $\Rightarrow$ Ánh sáng tới là Sub-Poisson 
	\item  Nếu ánh sáng tới có thống kê Poisson với $(\Delta n)^{2} = \overline{n} \rightarrow (\Delta N)^{2} = \eta \overline{n} = \overline{N}$ với mọi giá trị $\eta$. Nói cách khác, thống kê photon đếm được luôn cho phân bố Poisson\\
	\item  Nếu $\eta \ll 1$, sự thăng giáng của số photon đếm được có xu hướng tiến tới kết quả Poisson với $(\Delta N)^{2} = \eta \overline{n} = \overline{N}$ bất chấp photon tuân theo thống kê nào. 
\end{itemize}
Như vậy, để có thể quan sát được Sub-Poisson của photon thì buộc phải có $\eta \approx 1$ tức là hiệu suất gần 100\%.
\section{Shot noise in photodiodes}
Cách đếm photon ta vừa khảo sát chỉ phù hợp cho cường độ nhỏ $10^{6}$ photon mỗi giây hoặc ít hơn. Với những chùm mạnh hơn, detector không xử lý kịp thì cần phải tìmm cách khác.\\
Cách thường dùng: \textbf{photodiode detectors}. Photodiode là thiết bị bán dẫn tạo ra dòng điện trong mạch ngoài khi photon kích thích electron từ dải hoá trị lên dải dẫn. Một thông số quan trong của photodiode là hiệu suất lượng tử $\eta$, được đinhj nghĩa trong trường hợp này là tỉ lệ giữa số electrong quang được tạo ra trong mạch ngoài với số photon tới. Do đó, dòng điện được tạo ra trong mạch ngoài cho một dòng photon tới $\phi$, tức là dòng quang điện $i$, được định nghĩa
\begin{gather}
	i = \eta e \Phi = \eta e \f{P}{\hbar \omega}
\end{gather}
trong đó $e$ là điện tích nguyên tố, $P$ là công suất chùm tia, và $\omega$ là tần số góc. Tỉ số $\f{i}{P} = \f{\eta e}{\hbar \omega}$ được gọi là \textbf{độ nhạy} của photodiode có thứ nguyên là $A/W$. Giá trị của $\eta$ có thể được tính toán từ độ nhạy do được ở bước sóng phát hiện. (Những photodiodes tốt có thể đạt hiệu suất lượng tử $\eta = 90\%$). Nguyên lý đằng sau việc sử dụng các bộ dò photodiode để nghiên cứu các tính chất thống kê của ánh sáng là dòng quang điện được tạo ra bởi chùm sáng sẽ thăng giáng do sự thăng giáng cơ bản trong số lượng photon tới, được phản ánh bằng trong sự thăng giáng của dòng quang điện với độ tin cậy được xác định bởi $\eta$. Sự thăng giáng này được biểu hiện dưới dạng \q{noise} trong dòng photon
\image{noise.png}
dòng quang điện($i(t)$) biến thiên theo thời gian có thể được phân tách thành một thành phần dòng điện trung bình không phụ thuộc thời gian là $\mean{i(t)}$ và một thành phần dao động theo thời gian gọi là $\Delta i(t)$ theo đó ta có 
\begin{gather}
	i(t) = \mean{i} + \Delta i(t)
\end{gather}
Giá trị trung bình của $\Delta i$ tất nhiên là phải bằng không, nhưng giá trị trung bình của bình phương $\Delta i$ cụ thể là $\mean{\left(\Delta i (t)\right)^{2}}$ phải khác không. Vì dòng quang điện chảy qua điện trở tải $R_{L}$, sinh ra năng lượng với tốc độ $i^{2} R_{L}$, nên thuận tiện khi phân tích các dao động dưới dạng công suất nhiễu và biến thiên theo thời gian
\begin{gather}
	P_{\text{noise}}(t) = (\Delta i(t))^{2} R_{L}
\end{gather} 
Giả sử ta chiếu nguồn laser một mode vào photodiode. Vì ta đang xét nguồn gần như kết hợp hoàn hảo, tuân theo phân bố Poisson. Khi đó, thống kê quang điện tử cũng tuân theo phân bố này, với
\begin{gather}
	(\Delta N)^{2} = \mean{N}
\end{gather}
vì $i(t)$ tỉ lệ với số quang điện tử được tạo ra mỗi giây, khi đó độ lệch chuẩn của dòng photon thoả
\begin{gather}
	(\Delta i)^{2} \propto \mean{\Delta i}
\end{gather}
Biến đổi Fourier cho $i(t)$ và độ lệch chuẩn của sự thăng giáng trong dòng photon với độ rộng dải $\Delta f$, ta có
\begin{gather}
	(\Delta i)^{2} = 2e \Delta f \mean{i}
\end{gather}
từ đó suy ra
\begin{gather}
	P_{\text{noise}}(f) = 2 e R_{L} \Delta f \mean{i}
\end{gather}
Từ hai phương trình, ta hình dung được đại lượng \textbf{shot noise} với hai tính chất
\begin{itemize}
	\item Biến thiên(độ lệch chuẩn) của các dao động dòng điện(hoặc tương đương, công suất nhiễu) tỉ lệ thuận trực tiếp với giá trị trung bình của dòng điện
	\item Trường hợp khi phổ nhiễu \q{trắng}, tức là không phụ thuộc vào tần số
\end{itemize}
Đặc điểm thứ hai là hệ quả của thời gian ngẫu nhiên giữa các photon đến trong một chùm tia có thống kê Poisson. Tính \q{trắng} của nhiễu phụ thuộc vào thơi gian đáp ứng $\tau_D$ của photodiode, trong thực tế shot noise chỉ có thể được phát hiện lên đến tần số tối đa xấp xỉ $\approx \frac{1}{\tau_{D}}$ hình (b). 
\subsection{Sub-shot-noise photocurrent}
Ta có thể định lượng shot-noise redution bằng \textbf{nhân tử Fano}
\begin{gather}
	F_{\text{Fano}} = \f{\text{measured noise}}{\text{shot noise limit}}
\end{gather}
Nếu như tổng hiệu suất của hệ đường chuẩn sang dòng điện từ pin tới quang điện trong mạch detector là $\eta_{\text{total}}$ thì hệ số Fano được mong muốn là
\begin{gather}
	F_{\text{Fano}} = \eta_{\text{total}} F_{\text{drive}} + (1 - \eta_{\text{total}})
\end{gather}
\section{Bài tập}
\subsection*{Chứng minh rằng $(\Delta n)^2 = \dps\sum_{n} (n - \overline{n}) P_{\omega}(n) = \overline{n} + \overline{n}^2$}
Giải: \\
Đặt $x \equiv \exp(-\f{\hbar \omega}{k_B T})$, mà từ công thức (5.20) Mark Fox, ta biết $P(\omega)$ có dạng
\begin{align*}
	P_{\omega}(n)
	& = \f{\exp(-E_{n} / k_B T)}{\sum_{n = 0}^{\infty}(\exp(-E_{n} / k_B T)}                    \\
	& = \f{\exp(-n \hbar \omega / k_B T)}{\sum_{n = 0}^{\infty}(\exp(-n \hbar \omega / k_B T)}, \\
	& = \f{x^{n}}{\sum_{n = 0}^{\infty} x^{n}}. \tag{1}
\end{align*}
Ta xét chuỗi hình học
\begin{align*}
	\sum_{n = 0}^{\infty} x^{n} = \f{1 - x^n}{1 - x} \tag{2},
\end{align*}
khi $x<1$, thì (2) trở thành
\begin{align*}
	\sum_{n = 0}^{\infty} x^{n} = \f{1}{1 - x}, \tag{3}
\end{align*}
dẫn đến ta có thể viết lại $P_{\omega}(n)$
\begin{align*}
	P_{\omega}(n)
	& = x^{n} (1-x)                                                                               \\
	& \equiv \left( 1 - \exp(-\hbar \omega / k_B T) \right) \exp(-n\hbar \omega / k_B T). \tag{4}
\end{align*}
Số photon trung bình được cho bởi
\begin{align*}
	\overline{n}
	& = \sum_{n = 0}^{\infty} n P_{\omega}(n)                                                                                       \\
	& = \sum_{n = 0}^{\infty} n x^{n} (1-x)                                                                                         \\
	& = (1 - x) x \f{d}{dx} \left( \sum_{n = 0}^{\infty} x^{n} \right) (\text{Tại vì đạo hàm thì số bậc của $x$ là $n - 1$})        \\
	& = (1 - x) x \f{d}{dx} \left(\f{1}{1 - x}\right)                                                                               \\
	& = (1 - x) x \f{1}{(1 - x)^2}                                                                                                  \\
	& = \f{x}{1 - x}                                                                                                                \\
	& = \f{\exp(-\f{\hbar \omega}{k_B T})}{1 - \exp(-\f{\hbar \omega}{k_B T})} = \f{1}{\exp(\f{\hbar \omega}{k_B T}) - 1} . \tag{5}
\end{align*}
Phương trình (5) dẫn đến được
\begin{align*}
	x = \f{\overline{n}}{\overline{n} + 1},
\end{align*}
ta thay vô phương trình $P_{\omega}(n) = x^{n} (1-x)$, và được
\begin{align*}
	P_{\omega}(n)
	& = \left(1 - \f{\overline{n}}{\overline{n} + 1}\right) \left(\f{\overline{n}}{\overline{n} + 1}\right)^{n} \\
	& = \left(\f{1}{\overline{n} + 1}\right)\left(\f{\overline{n}}{\overline{n} + 1}\right)^{n} .
\end{align*}
Phương sai Var($n$) $\equiv \Delta n$
\begin{align*}
	(\Delta n)^2
	& = \sum_{n = 0}^{\infty}(n - \overline{n})^2 P_{\omega}(n)                                                                                                                                                                  \\
	& = \sum_{n = 0}^{\infty}(n^2 - 2n\overline{n} + \overline{n}^2) x^{n} (1-x)                                                                                                                                                 \\
	& = \sum_{n = 0}^{\infty}n^2 x^{n} (1-x) - \sum_{n = 0}^{\infty}2n\overline{n} x^{n} (1-x) + \sum_{n = 0}^{\infty}\overline{n}^2 x^{n} (1-x)                                                                                 \\
	& = (1-x)\sum_{n = 0}^{\infty}n^2 x^{n} - 2\overline{n}(1-x)\underbracket{\sum_{n = 0}^{\infty}n x^{n}}_{\frac{x}{(1 - x)^2}}  +\, \overline{n}^2(1-x) \underbracket{\sum_{n = 0}^{\infty} x^{n}}_{\frac{1}{1 - x}}. \tag{6}
\end{align*}
Xét số hạng đầu tiên $(1-x)\sum_{n = 0}^{\infty}n^2 x^{n}$, ta có
\begin{align*}
	\f{d}{dx} \left( \sum_{n = 0}^{\infty} x^{n} \right)
	& = \sum_{n = 0}^{\infty} n x^{n-1}                          \\
	\Rightarrow \f{d^2}{dx^2} \left( \sum_{n = 0}^{\infty} x^{n} \right)
	& = \f{d}{dx} \left( \sum_{n = 0}^{\infty} n x^{n-1} \right) \\
	\f{d^2}{dx^2} \left( \sum_{n = 0}^{\infty} x^{n} \right) & = \sum_{n = 0}^{\infty} n(n-1) x^{n-2}                     \\
	\f{d^2}{dx^2} \left( \sum_{n = 0}^{\infty} x^{n} \right) & = \sum_{n = 0}^{\infty} (n^2-n) x^{n-2}. \tag{7}
\end{align*}
Nhân 2 vế của (7) cho $x^2$ ta được
\begin{align*}
	x^2 \f{d^2}{dx^2} \left( \sum_{n = 0}^{\infty} x^{n} \right)
	& = \sum_{n = 0}^{\infty} (n^2-n) x^{n}
	= \sum_{n = 0}^{\infty} n^2 x^{n} - \sum_{n = 0}^{\infty} nx^{n}                                                                           \\
	\Rightarrow \sum_{n = 0}^{\infty} n^2 x^{n}
	& = x^2 \f{d^2}{dx^2} \left( \sum_{n = 0}^{\infty} x^{n} \right) + \sum_{n = 0}^{\infty} nx^{n}                                           \\
	& = x^2 \f{d^2}{dx^2} \left( \sum_{n = 0}^{\infty} x^{n} \right) + x \f{d}{dx} \left( \sum_{n = 0}^{\infty} x^{n} \right) .(DPCM) \tag{8}
\end{align*}
Thay vô (6) ta được
\begin{align*}
	(\Delta n)^2
	& = (1-x) \left[x^2 \f{d^2}{dx^2} \left( \sum_{n = 0}^{\infty} x^{n} \right) + x \f{d}{dx} \left( \sum_{n = 0}^{\infty} x^{n} \right)\right] - \f{2\overline{n}(1-x)x}{(1 - x)^2} + \overline{n}^2 \\
	& = (1-x) \left[x^2 \f{d^2}{dx^2} \left( \f{1}{1 - x} \right) + x \f{d}{dx} \left( \f{1}{1 - x} \right)\right] - \f{2\overline{n}(1-x)x}{(1 - x)^2} + \overline{n}^2                               \\
	& = (1-x) \left[\f{2x^2}{(1-x)^3} + \f{x}{(1 - x)^2}\right] - \f{2\overline{n}(1-x)x}{(1 - x)^2} + \overline{n}^2                                                                                  \\
	& = \left[2 \overline{n}^2 + \overline{n}\right] - 2 \overline{n}^2 + \overline{n}^2                                                                                                               \\
	& = \overline{n} + \overline{n}^2 (DPCM).\tag{9}
\end{align*}

\subsection*{Chứng minh rằng $\langle \Delta E^2 \rangle = k_B T^2 \f{\partial \langle E \rangle}{T}$}
Ta có:
\begin{align*}
	\langle \Delta E^2 \rangle = \langle E^2 \rangle - \langle \Delta E \rangle^2.
\end{align*}
Xét:
\begin{align*}
	\langle E \rangle 
	&= \sum_{n} p(n) E_{n} \\
	&= \f{1}{Z} \sum_{n} E_{n} e^{-\beta E_{n}} \\
	&= \f{1}{Z} \left( -\f{\partial}{\partial \beta} \sum_{n}e^{-\beta E_{n}} 
	\right)\\
	&= \f{1}{Z} \left( -\f{\partial}{\partial \beta} Z \right),
\end{align*}
với $p(n) = \f{e^{-\beta E_{n}}}{Z}$ là xác suất hệ được tìm thấy ở trạng thái $\ket{n}$. \\
Xét
\begin{align*}
	\langle E^2 \rangle
	&= \sum_{n} p_{n} E_{n}^{2} \\
	&= \f{1}{Z} \sum_{n} E_{n}^{2} e^{-\beta E_{n}} \\
	&= \f{1}{Z} \left( \f{\partial^2}{\partial \beta^2} \sum_{n} E_{n}^{2} \right) \\
	&= \f{1}{Z} \left( \f{\partial^2}{\partial \beta^2} Z\right),
\end{align*}
nên ta có:
\begin{align*}
	\langle \Delta E^2 \rangle 
	&= \langle E^2 \rangle - \langle \Delta E \rangle^2 \\
	&= \f{1}{Z} \left( \f{\partial^2 Z}{\partial \beta^2} \right) - \f{1}{Z^2} \left( \f{\partial Z}{\partial \beta} \right)^2 \\
	&= \f{Z^{''}}{Z} - \f{Z^{'2}}{Z^2} \\
	&= \left(\f{Z^{'}}{Z}\right)^2 \\
	&= \left[ (\ln Z)^{'} \right]^{'} \\
	&= \left( \ln Z \right)^{''} \\
	&= \f{\partial^2}{\partial \beta^2} \ln Z = -\f{\partial \langle E \rangle}{\partial \beta}.
\end{align*}
mà
\begin{align*}
	\langle E \rangle = -\f{\partial}{\partial Z} \ln Z \quad \quad \text{với}\; \beta = \f{1}{k_{B}T}.
\end{align*}
Vậy
\begin{align*}
	\langle \Delta E^2 \rangle = -\f{\partial \langle E \rangle}{\partial \left( \f{1}{k_{B}T}\right)} = \f{-\partial \langle E \rangle}{-\frac{k_{B}}{k_{B}T}\partial T} = k_{B} T^2 \f{\partial \langle E \rangle}{\partial T}
\end{align*}


\subsection*{Chứng minh rằng $D(\omega) = \frac{\omega^2}{\pi^2 c^3}$}
Giải:\\
Mỗi trạng thái $\mathbf{k}$ trong mạng có 1 cặp trạng thái 1 hạt.
\begin{align*}
	\begin{cases}
		\text{2 trạng thái chiếm thể tích :} \; \left(\f{2\pi}{L}\right)^3 \\
		\text{? số trạng thái chiếm thể tích :} \; d\mathbf{k}             \\
	\end{cases}
\end{align*}
số trạng thái trong $d\mathbf{k}$ là
\begin{align*}
	D(k)dk = 2  \left(\f{L}{2\pi}\right)^3 d\mathbf{k} = \f{2V}{(2\pi)^3} 4\pi k^2 dk = \f{V}{\pi^2}k^2,
\end{align*}
mà
\begin{align*}
	D(\omega) d\omega     & = D(k) dk                 \\
	\Rightarrow D(\omega) & = \f{D(k)}{d\omega / dk}.
\end{align*}
Đặt $\omega = ck$ ta có được
\begin{align*}
	g(\omega) = \f{\omega^2}{\pi^2 c^3} (DPCM).
\end{align*}

\subsection*{Bài tập trong Mark Fox}
\subsubsection*{5.1}
Ta có bước sóng $\lambda = 633$ nm, với công suất là $P = 0.01$ pW, với hiệu suất lượng tử là $\eta = 30\% $ với khoảng thời gian là $T = 10$ ms
\begin{enumerate}
	\item[(a)] Tốc độ đếm $\mathcal{R} = \f{\eta P}{\hbar \omega} = 0.3 \f{0.01 \times 10^{-12}}{\hbar \frac{2\pi c}{\lambda}} = 9553.174$ count/s.
	\item[(b)] Tốc độ đếm trung bình $N(T)= \eta \f{PT}{\hbar \omega} = 0.3 \times \mathcal{R}\times 0.01 =$
	\item[(c)] Ta đã giả sử rằng photon detected count có thống kê Poissonian, và đã biết là $\overline{n} = N$, do đó $\Delta n = \sqrt{\overline{n}}$.
\end{enumerate}
\subsubsection*{5.4}
Tính số photon trung bình cho từng mode tại nguồn là đèn Tungsten có bước sóng $\lambda =$ 500 nm, tại $T = $ 2000 K là nhiệt độ cần để có được $\overline{n} = 1$ tại bước sóng $\lambda$. Nhiệt độ tại bước sóng là $\lambda^{'} = 10 \mu$m. Từ công thức Plank
\begin{align*}
	\overline{n} = \f{1}{\exp(\frac{\hbar \omega}{k_{B} T}) - 1} = \frac{1}{\exp(\frac{hc}{\lambda k_{B} T}) - 1} = \f{1}{\exp(\frac{hc}{500\times 10^{-9} 1.38 \times 10^{-23} 2000}) - 1} \approx 5.548 \times 10^{-7}. %\text{photon}
\end{align*}
Để $\overline{n} = 1$ thì $T$ là bao nhiêu?
\begin{align*}
	\overline{n} = 1 = \f{1}{\exp(\frac{hc}{500\times 10^{-9} 1.38 \times 10^{-23} T}) - 1} \Rightarrow T = 41562.6 \text{K}.
\end{align*}
Với $\lambda = \lambda'$ và để $\overline{n} = 1$ thì $T$ là bao nhiêu?
\begin{align*}
	\overline{n} = 1 = \f{1}{\exp(\frac{hc}{10 \times 10^{-6} 1.38 \times 10^{-23} T}) - 1} \Rightarrow T = 2078.13 \text{K}.
\end{align*}
\subsubsection*{5.5}
\subsubsection*{5.8}
Một chùm tia với thông lượng photon là $\Phi = 1000$ photon s$^{-1}$ tới detector với hiệu suất lượng tử là $\eta = 20 \%$. Nếu khoảng thời gian được xét là $T = 10$s, tính trung bình và độ lệch chuẩn của $N(t)$ trong các trường hợp sau:
\begin{enumerate}
	\item[(a)] Ánh sáng có thống kê Poissonian
	\item[(b)] Ánh sáng có Super-Poissonian với độ lệch chuẩn $\Delta n = 2\times \Delta n_{\text{Poissonian}}$
	\item[(c)] Ánh sáng là trạng thái số photon
\end{enumerate}
Giải:
Số photon trung bình va chạm vào detector:
\begin{align*}
	\overline{n} = \Phi T = 10^4.
\end{align*}
Tốc độ đếm trung bình
\begin{align*}
	\overline{N} = \eta \overline{n} = 2\times 10^3.
\end{align*}
\begin{enumerate}
	\item[(a)] Ánh sáng có thống kê Photon, Average photon count number $N(t)$ là
	\begin{align*}
		\overline{N}(t) \equiv \overline{n}= \eta \Phi T = 0.2 \times 1000 \times 10 = 2000 \; \text{photon s}^{-1}.
	\end{align*}
	Độ lệch chuẩn $(\Delta \overline{N})$
	\begin{align*}
		(\Delta \overline{N})^2 = \overline{n} = \overline{N} \Rightarrow \Delta \overline{N} = \sqrt{\overline{N}} = \sqrt{2000}.
	\end{align*}
	\item[(b)] Ánh sáng có Super-Poissonian\\
	Số photon trung bình (Average/mean photon count number) là $\overline{N}(t) = 2000$. Ta có
	\begin{align*}
		(\Delta \overline{N})^2
		& = \eta^2 (\Delta n)^2 + \eta ( 1 - \eta ) \overline{n} \tag{5.56 M.Fox}                                                                     \\
		& =  \eta^2 \times (2 \times \Delta n_{\text{Poissonian}})^2 + \eta ( 1 - \eta ) \overline{n}                                                 \\
		& = 0.2^2 \times 4 \times \overline{n} + 0.2 \times (1 - 0.2) \times \overline{n} = 3200 \Rightarrow \Delta N = \sqrt{3200}. \tag{5.15 M.Fox}
	\end{align*}
	trong đó $\Delta N$ là phương sai của photoncount number,
	\item[(c)]
	Độ lệch chuẩn $(\Delta \overline{N})$
	\begin{align*}
		(\Delta \overline{N})^2 = \eta(1-\eta) \overline{n} = \sqrt{1600}.
	\end{align*}
\end{enumerate}

\subsubsection*{5.9}
Cường độ dòng photon ($i$) là
\begin{align*}
	i = \eta e \f{P}{\hbar \omega} = 0.9 \abs{1.6\times 10^{-19}} \f{10 \times 10^{-3}}{\hbar c / \lambda} = 4.584 \text{A},
\end{align*}
noise power đi qua 1 đơn vị băng thông là
\begin{align*}
	P_{\text{noise}} = 2e \times 50 \times 4.584  = 458.4e \; \text{W Hz}^{-1}.
\end{align*}
\subsubsection*{5.12}
Xét dòng điện đi qua điện trở $R$, tại nhiệt độ $T$ K trong mạch điện trở. Sự thăng giáng của dòng điện bên trong giải tần số $\Delta f$ được cho bởi hệ thức Johson noise:
\begin{align*}
	\langle (\Delta i)^2 \rangle = 4 k_{B} T \f{\Delta f}{R} .
\end{align*}
Chứng minh rằng Johson shot noise cho kết quả nhỏ hơn shot noise với cùng một giá trị trung bình dòng điện được cho bởi điện thế đi qua điện trở là lớn hơn $\f{2k_B T}{e}$, và tính điện thế này với $T = 300$K.\\
Giải:\\
Ta có phương sai cường độ dòng điện $\Delta i$ cho shot noise là
\begin{align*}
	(\Delta i)_{\text{shot noise}}^2
	& = 2e \Delta f \langle i \rangle                                 \\
	& = 2e \Delta f \f{V}{R}                                          \\
	& \equiv \langle (\Delta i)^2 \rangle_{\text{shot noise}} \tag{1}
\end{align*}
Hệ thức Johnson cho
\begin{align*}
	\langle (\Delta i)^2 \rangle_{\text{Johnson}} = 4 k_{B} T \f{\Delta f}{R}. \tag{2}
\end{align*}
Để (2) < (1) thì
\begin{align*}
	4 k_{B} T \f{\Delta f}{R} & < 2e \Delta f \f{V}{R}           \\
	\Rightarrow V             & > \f{2k_B T}{e}  \; \text{ĐPCM}.
\end{align*}
Điện thế này tại $T = 300$K
\begin{align*}
	V > \f{2k_B T}{e} = \f{2 \times 1.38 \times 10^{-23} \times 300}{1.6 \times 10^{-19}} = 0.05175 \text{V}.
\end{align*}

\subsubsection*{5.13}
Hiệu suất lượng tử của một bóng đèn LED được định nghĩa là tỉ số giữa số photon phát xạ so với số electron chạy qua máy dò. Một đèn LED phát ra ánh sáng ở tần số 800 nm, được định hướng bởi nguồn là một ắc quy có điện thế là 9V thông qua điện trở $R = 1000 \, \Omega$. Bóng đèn LED có hiệu suất lượng tử $\eta_{\text{LED}} = 40\%$, và có $\eta_{1} = \eta_{\text{optic}} = 80\%$ photon phát xạ được tập trung vào một máy dò photodiode với hiệu suất lượng tử là $\eta_{2} =\eta_{\text{Detect}} = 90\%$.
\begin{enumerate}
	\item[(a)] Năng lượng photon là
	\begin{align*}
		E_{\text{photon}} = \f{\hbar 2 \pi c}{\lambda} = 1.55eV.
	\end{align*}
	Biết rằng điện thế lên đèn LED là xấp xỉ với năng lượng photon, tính bằng eV trong điều kiện hoạt động bình thường, nên ta có:
	\begin{align*}
		U_{\text{LED}} = 1.55 V.
	\end{align*}
	Điện thế đi qua điện trở là:
	\begin{align*}
		U_{\text{nguồn}} = U_{\text{LED}} + U_{R} = 9 V \Rightarrow U_{R} = 9 - 1.55 = 7.45 \text{V}.
	\end{align*}
	Dòng điện đi qua $U_{R}$
	\begin{align*}
		I_{R} = \f{U_{R}}{R} = 7.45 \text{mA}.
	\end{align*}
	\image{circuit.png}
	Do không có sự thăng giáng nên trung bình dòng điện sẽ bằng dòng điện đi qua $U_{R}$
	\begin{align*}
		i_{1} = I_{R} = 7.45 \text{mA}.
	\end{align*}
	\item[(b)] Ta có $T = 293 K$\\
	Hệ số $F_{\text{Fano}}$ được cho bởi
	\begin{align*}
		F_{\text{Fano}}
		& = \f{\text{measured noise}}{\text{shot noise limit}}                  \\
		& = \f{(\Delta i)_{\text{Johnson}}^2}{(\Delta i)_{\text{shot noise}}^2} \\
		& = \f{4 k_B T \Delta f}{2e R \Delta f \langle i \rangle}               \\
		& = 6.77 \times 10^{-3}
	\end{align*}
	\item[(c)] Dòng photon trung bình được tính bởi
	\begin{align*}
		\langle i \rangle_{\text{photon}} = \eta e \Phi \tag{5.58 \text{M.Fox}}
	\end{align*}
	trong đó $\eta$ là hiệu suất lượng tử của photodiode detector, và $\Phi$ là thông lượng photon đi tới PD\\
	\begin{align*}
		\Phi = \f{P_{\text{LED}}}{E_{\text{photon}}}
	\end{align*}
	Ta viết lại (5.58)
	\begin{align*}
		\langle i \rangle_{\text{photon}}
		& = N_{\text{photonelectron}} \times e                                  \\
		& = \eta_{\text{Detec}} e \Phi                                          \\
		& = \eta_{\text{Detec}} e \f{i_{1} V_{\text{nguồn}}}{E_{\text{photon}}} \\
		& = \eta_{\text{Detec}} e \f{i_{1}}{e} \eta_{\text{LED}} \eta_{2}       \\
		& = 2.145 \text{mA}
	\end{align*}
	\item[(d)] Tổng hiệu suất lượng tử
	\begin{align*}
		\eta_{\text{tolta}} = \eta_{\text{Detec}} \times \eta_{\text{LED}} \times \eta_{\text{optic}} = 0.288  \tag{*}
	\end{align*}
	(*) Bachor, HA., Rottengatter, P. $\&$ Savage, C.M. Correlation effects in light sources with high quantum efficiency. Appl. Phys. B 55, 258–264 (1992).\\ https://doi.org/10.1007/BF00325014 \\
	Hệ số $F_{\text{Fano}}$ được cho bởi
	\begin{align*}
		F_{\text{Fano}} = \eta_{\text{tolta}} F_{\text{drive}} + (1 - \eta_{\text{tolta}})
		& = 0.288 \times 6.77 \times 10^{-3} + (1 - 0.288) \\
		& = 0.7139
	\end{align*}
	\item[(e)]
	\begin{align*}
		P_{\text{shot noise}}
		& = 2 e R_{L} \Delta f \langle i \rangle_{\text{photon}}            \\
		& = 2e \times 50 \times 10 \times 10^{3} \times 2.15 \times 10^{-3} \\
		& = 3.44 \times 10^{-16} \text{W} \tag{1}
	\end{align*}
	\begin{align*}
		F_{\text{photon}} = \f{P_\text{measured}}{P_{\text{photon}} \equiv P_{\text{shot noise}} } = 0.714 \Rightarrow P_\text{measured}
		& = 0.714 \times 3.44 \times 10^{-16} \\
		& = 2.45 \times 10^{-16} \text{W}
	\end{align*}
\end{enumerate}

\chapter{Buổi 7}
\section{Light waves as classical harmonic oscillators}
\begin{gather}
	\text{Light(wave)} \leftrightarrow \text{H.O}
\end{gather}
\subsection{Harmonic oscillators}
\begin{equation}
	\begin{aligned}
		&\ddot{x} + \omega^{2} x = 0 \\
		&\rightarrow x = x_{0} \sin(\omega t)\\
		&\Rightarrow p_{x}(t) = m \dot{x} = p_{0} \cos(\omega t) \quad \text{với}\; p_{0} = m \omega x_{0}
	\end{aligned}
\end{equation}
Đổi sang hệ toạ độ chuẩn tắc: $(x,p_{x}) \rightarrow (q,p)$ với $q \equiv \sqrt{m}x$
\subsection{Light}
Ta chọn chiều là theo trục $z$ là chiều truyền của ánh sáng
\begin{gather}
	E_{x} (z, t) = E_{0} \sin (kz) \sin (\omega t)
\end{gather}
$\rightarrow$ đã chọn pha $\varphi = 0$ $\Rightarrow$ Tổng quát $\sin (\omega t + \varphi)$
Tổng quát
\begin{gather}
	\vv{E}(\vv{r},t) = \vv{\epsilon} \mathcal{E}(t) e^{i \mathbf{k \cdot r}} + c.c
\end{gather}
trong đó
\begin{itemize}
	\item $\vv{\epsilon}$: Vecto phân cực(vector đơn vị), và vuông góc với $\vv{k}$
	\item $\mathcal{E}$: là số phức, biên độ phụ thuộc vào $\omega , \varphi$
\end{itemize}
từ phương trình Maxwell
\begin{gather}
	\f{\partial B_{y}}{\partial z} = \varepsilon_{0} \mu_{0} \f{\partial \mathcal{E}_{x}}{\partial t}
\end{gather}
ta có
\begin{gather}
	B_{y}(z,t) = B_{0} \cos(k z) \cos (\omega t)
\end{gather}
với $B_{0} = \f{E_{0}}{c}$. Mật độ năng lượng
\begin{gather}
	U = U_{E} + U_{B} = \f{1}{2}  \left( \epsilon \mathcal{E}^{2} + \f{1}{\mu_{0}B^{2}} \right)
\end{gather}
Năng lượng điện trường
\begin{equation}
	\begin{aligned}
		E_{\text{electric}}
		&= \f{1}{2} \epsilon_{0} A \int_{0}^{L} \mathcal{E}_{0}^{2} \sin^{2} kz \sin^{2} \omega t dz\\
		&= \f{1}{4} A \mathcal{E}_{0}^{2} \sin^{2} \omega t \int_{0}^{L} (1 - \cos 2 kz)dz\\
		&= \f{1}{4} \epsilon_{0} V \mathcal{E}_{0}^{2} \sin \omega t
	\end{aligned}
\end{equation}
trong đó $V - LA$, điều kiện biên $\sin kL = 0$. Điều kiện biên chính xác tại các bức tường của buồng cộng hưởng là không quan trọng, vì chúng có thể chỉ bổ sung một hạng tử thay đổi theo $1/L$ khi thực hiện phép tích phân theo $z$, và hạng tử này có thể bị bỏ qua khi $L$ đủ lớn. Việc không nhạy cảm với các điều kiện biên là quan trọng vì chúng ta không muốn bất kỳ kết quả cơ bản nào phụ thuộc vào sự có mặt của buồng cộng hưởng. Năng lượng từ trường
\begin{gather}
	E_{\text{magnetic}} = \f{1}{4} \mu_{0} V B_{0}^{2} \cos \omega t
\end{gather}
Ta có thể viết năng lượng điện từ thành
\begin{gather}
	E = \f{V}{4} (\epsilon \mathcal{E}_{0} \sin^{2} \omega t + \f{ B_{0}^{2} }{ \mu_{0} } \cos^{2} \omega t )
\end{gather}
Thay $(E,B) \Rightarrow (q,p)$
\begin{gather}
	q(t) = \sqrt{\f{\epsilon_{0} V}{2\omega}} \mathcal{E}_{0} \sin \omega t\\
	p(t) = \sqrt{\f{V}{2 \mu_{0} }} B_{0} \cos \omega t
\end{gather}
và ta có
\begin{gather}
	p = \dot{q}\\
	\dot{p} = - \omega^{2} q
\end{gather}
Điều này cho ra kết quả tương tự H.O
\section{Giản đồ pha, trường vuông góc}
Ta xét thêm thành phần pha(phụ thuộc cách chọn gốc thời gian $t=0$)
\begin{equation}
	\begin{aligned}
		\mathcal{E}_{x}(z,t)
		&= \mathcal{E}_{0} \sin(kz) \sin(\omega t + \varphi)\\
		&= \mathcal{E}_{1} \sin(\omega t) + E_{2} \cos(\omega t)\\
		&= \mathcal{E}_{1} \cos(\omega t - \f{\pi}{2}) + E_{2} \cos(\omega t)
	\end{aligned}
\end{equation}
\image{phaseDiag.png}
trong đó $E_{1}$ và $E_{2}$ được gọi là trường vuông góc do pha lệch nhau góc $\pi / 2$. Ta có thể viết lại phương trình trên thành 
\begin{gather}
	\mathcal{E}(z) = \mathcal{E}_{1} + i \mathcal{E}_{2}
\end{gather}
trường vuông góc $X_{1}$ và $X_{2}$ được xác định như sau
\begin{gather}
	X_{1} = \sqrt{\f{\epsilon_{0} V}{4 \hbar \omega}} \mathcal{E}_{0} \sin (\omega t)\\
	X_{2} = \sqrt{\f{\epsilon_{0} V}{4 \hbar \omega}} \mathcal{E}_{0} \cos (\omega t)
\end{gather}
Điện trường, từ trường được viết theo $X_{1}$ và $X_{2}$ là
\begin{gather}
	E_{x}(z, t) = \sqrt{\f{4 \hbar \omega}{\epsilon_{0} V}} \sin kz \left[ \cos \phi X_{1} + \sin \phi X_{2} \right]\\
	B_{y}(z, t) = \f{1}{c} \sqrt{\f{4 \hbar \omega}{\epsilon_{0} V}} \cos kz \left[ \cos \phi X_{2} - \sin \phi X_{1} \right]
\end{gather}
Công thức liên hệ giữa trường vuông và toạ độ, động lượng chính tắc
\begin{gather}
	X_{1}(t) = \sqrt{\f{\omega}{2 \hbar}} q(t)\\
	X_{2}(t) = \sqrt{\f{1}{2\hbar \omega}}p(t)
\end{gather}
Thay vì dùng $E,B$ để biểu diễn sóng điện từ, ta biểu diễn thông qua $X_{1}$ và $X_{2}$.
\section{Xem ánh sáng như là dao động điều hoà lượng tử}
\begin{itemize}
	\item Năng lượng được lượng tử hoá theo hệ số $\hbar \omega$ 
	\begin{gather}
		E_{n} = \left(n + 1/2\right) \hbar \omega
	\end{gather}
	Nhưng nếu $n = 0$($n$ là số photon chứ không phải mức trạng thái), không hề có suất hiện photon nhưng vẫn có năng lượng là $E_{0} = \hbar \omega /2$ $\Rightarrow$ Điều này là không có trong cổ điển.
	\item Nguyên lý bất định
	\begin{gather}
		\Delta x \Delta p \geq \hbar /2
	\end{gather}
	Ta có
	\begin{equation}
		\begin{aligned}
			\Delta X_{1} \Delta X_{2}
			&= \sqrt{\f{\omega}{2 \hbar}} \Delta q \sqrt{\f{1}{2 \hbar \omega}} \Delta p_{x} \\
			&= \f{1}{2 \hbar} \Delta q \Delta p_{x}\\
			&= \f{1}{2 \hbar} \f{\Delta x}{\sqrt{m}} \sqrt{m} \Delta p_{x} \geq 1/4
		\end{aligned}
	\end{equation}
	Vì $X_{1},X_{2}$ tuân theo nguyên lý bất định nên cả $X_{1},X_{2}$ sẽ có độ bất định của riêng nó, gộp chung với nhau diễn tả độ bất định của $\Delta X_{1} \Delta X_{2}$. Bất định dẫn tới bất định về pha $\Delta \phi$ liên quan tới số hạt 
\end{itemize}
\image{phaseUncer.png}
\section{Tổng quan}
\begin{enumerate}
	\item[1.] \textbf{Cổ điển}\\
	Xét trường hợp tổng quát của điện trường cổ điển
	\begin{gather}
		\vv{E}(\vv{r},t) = \vv{\epsilon} \mathcal{E}(t) e^{i \mathbf{k \cdot r}} + \text{c.c}
	\end{gather}
	Thường ta chọn $\mathcal{E}$ có dạng như sau
	\begin{gather}
		\mathcal{E} = \mathcal{E}^{1} e^{-i (\omega t + \phi)}
	\end{gather}
	trong đó $\mathcal{E}^{1}$ là đơn mode, có nhiều mode khác nhau nhưng ta đang xét 1 mode. Tương tự cho từ trường cổ điện
	\begin{gather}
		\vv{B}(\vv{r},t) = \f{\vv{\epsilon} \times \vv{k}}{\omega} \mathcal{E}(t) e^{i \mathbf{k \cdot r}} + \text{c.c}
	\end{gather}
\end{enumerate}
\section{Fock state}
Là trạng thái có số hạt là xác định $\ket{n}$
\begin{itemize}
	\item $\mean{n} = \bra{n} \hat{N} \ket{n} = n$
	\item $\mean{n^{2}} = \bra{n} \hat{N}^{2} \ket{n} = n^{2}$ 
	\item $\Delta n = \sqrt{\mean{n^{2}} - \mean{n}^{2}} = 0$ $\Rightarrow$ Trthái Fock có số hạt là xác định
	\item $\mean{\hat{E}} = \bra{n} \hat{E} \ket{n} = 0$ Dù trạng thái Fock có số photon nhưng trung bình của giá trị vật lý $E$ bằng 0
	\item  $\mean{\hat{E}^{2}} = \bra{n} \hat{E}^{2} \ket{n} = (\mathcal{E}^{1})^{2}(2n + 1)$
	\item $\Delta \vv{E} = \sqrt{\mean{\hat{E}}^{2} - \mean{\hat{E}}^{2}} = \mathcal{E}^{1}\sqrt{2n + 1}$: độ thăng giáng của điện trường
\end{itemize}
trong đó $\hat{E}$ được định nghĩa là
\begin{gather}
	\hat{E}(r,t) = i \vv{\epsilon} \mathcal{E}^{1} \left[ \hat{a} e^{i(kr - \omega t)} - \hat{a}^{\dagger} e^{-i(kr - \omega t)} \right]
\end{gather}
Trạng thái Fock có số hạt xác định không mô tả được thông tin cho cổ điển\\
\textbf{Trạng thái chân không:} $\ket{0}(n = 0)$ Ta có
\begin{gather}
	\mean{\hat{N}}_{0} = 0 ; \quad E_{0} = \f{\hbar \omega}{2}\\
	\mean{\hat{E}}_{0} = 0 ; \quad \Delta \vv{E} = \mathcal{E}^{1}\\
	\Delta Q \Delta P = \f{1}{2}
\end{gather}
Trạng thái chân không là trạng thái có số hạt bằng 0 nhưng có năng lượng và độ thăng giáng. Nó lý giải cho cơ chế phát xạ tự phát, trong chân không tuy không có trường, không có hạt nhưng nó có thăng giáng chân không, nó vẫn có nền(nhiễu), chính cái đó tác động gây ra phát xạ tự phát. Trong trạng thái phát xạ tự phát, trnạg thái $\ket{n}$ có số hạt xác định thì sẽ không rơi xuống, hạt nằm im ở đó, xác định thì không thay đổi nên không có nhảy xuống, nhưng d ợ thăng giáng trong chân không như một nhiễu loạn gây ra sự thay đổi trạng thái. Người ta không thể đo được trực tiếp dao động chân không nhưng gián tiếp ta thấy như thế. Ngoài ra còn giải thích cho dịch chuyển lam và lực Casimir: ta có 2 tấm trung hoà điện nhưng đo thì vẫn có lực, ...
\section{Trạng thái coherent $\ket{\alpha}$}
Dùng trạng thái coherent có thể đo được được trường điện. $\alpha$ được xây dựng bởi trường vuông góc $X_{1}$ và $X_{2}$
\begin{gather}
	\alpha = X_{1} + i X_{2} = \abs{\alpha} e^{i \phi}
\end{gather}
trong đó
\begin{gather}
	\abs{\alpha} = \sqrt{X_{1}^{2} + X_{1}^{2}} = \sqrt{\f{\epsilon_{0}V}{4\hbar \omega}} \mathcal{E}_{0}\\
	X_{1} = \abs{\alpha} \cos \phi ; \quad X_{2} = \abs{\alpha} \sin \phi\\
	\Delta X_{1} = \Delta X_{2} = \f{1}{2} \rightarrow \text{độ bất định tối thiểu} 
\end{gather}
Khi $\mathcal{E}_{0}$ thì rơi vào trường hợp chân không(số hạt).\\
Trung bình điện trường
\begin{gather}
	E_{\text{classical}} = \hbar \omega \abs{\alpha}^{2}\\
	E_{\text{quantum}} = (\overline{n} + \frac{1}{2}) \hbar \omega
\end{gather}
$\rightarrow$ có thể dùng được cho cổ điển
\subsection{Mối liên hệ giữa trạng thái coherent và trạng thái Fock}
\begin{gather}
	\ket{\alpha} = e^{-\frac{\abs{\alpha}^{2}}{2}} \sum_{n = 0}^{\infty} \f{\alpha^{n}}{\sqrt{n!}} \ket{n}
\end{gather}
ngoài ra ta có
\begin{gather}
	\Delta n = \sqrt{\bra{\alpha}\hat{N}^{2}\ket{\alpha} - \bra{\alpha}\hat{N}\ket{\alpha}^{2}} = \sqrt{\overline{n}} \\
	\rightarrow \text{Trạng thái coherent, đúng như chờ đợi} \nonumber
\end{gather}
\section{Shot noise; bất định về số hạt và số pha}
Bất định về pha đi chung với số hạt
\begin{gather}
	\Delta n \Delta \phi \geq \f{1}{2}
\end{gather}
trong chương 5, coherent state cho ta ngưỡng shot noise là đường thẳng ngang, dưới là sub trên là super
\section{Squeezed states}
Bàn tới trạng thái bị bóp nén. Về nguyên tắc bóp đều là hình tròn. Một hướng là nén theo pha, một là nén theo biên độ, tuy nhiên diện tích sau khi nén vẫn phải được bảo toàn. Bất định vẫn xảy ra và vẫn phải là tối thiểu để điện tích không đổi.
\image{phavuong.png}
(a) là trạng thái chân không nén. (b) ánh sáng bị nén pha. (c) ánh sáng bị nén biên độ. Đường tròn nét đứt trong mỗi biểu đồ cho thấy độ bất định vuông pha của các trạng thái chân không hoặc trạng thái coherent với $\Delta X_{1} = \Delta X_{2} = \f{1}{2}$
\chapter{Buổi 8}
\section{Bài tập}
\subsection*{Bài tập 1}
\begin{align*}
	X_{1} (t) &= \left(\f{\omega}{2 \hbar}\right)^{1/2} q(t) \\
	X_{2} (t) &= \left(\f{1}{2 \hbar \omega}\right)^{1/2} p(t)
\end{align*}
ta biểu diễn toạ độ chính tắc và động lượng suy rộng dưới dạng $x$ và $p_{x}$
\begin{align}
	\begin{cases}
		X_{1} (t) &= \left(\f{\omega}{2 \hbar}\right)^{1/2} \sqrt{m} x(t) \\
		X_{2} (t) &= \left(\f{1}{2 \hbar \omega}\right)^{1/2} \f{1}{\sqrt{m}} p_{x}(t)
	\end{cases}
\end{align}	
ta có 
\begin{align*}
	a_{\pm} \equiv \f{1}{\sqrt{2 \hbar m \omega}} \left( \mp i p + m \omega x \right),
\end{align*}
và
\begin{align}
	\begin{cases}
		x = \sqrt{\f{\hbar}{2 m \omega}} \left( a_{+} + a_{-} \right)\\
		p = i \sqrt{\f{\hbar m \omega}{2}} \left( a_{+} - a_{-} \right)
	\end{cases}
\end{align}	
Thay (2) vào (1), ta được
\begin{align}
	\begin{cases}
		X_{1} (t) &= \left(\f{\omega}{2 \hbar}\right)^{1/2} \sqrt{m} \sqrt{\f{\hbar}{2 m \omega}} \left( a_{+} + a_{-} \right) \\
		X_{2} (t) &= \left(\f{1}{2 \hbar \omega}\right)^{1/2} \f{1}{\sqrt{m}}  i \sqrt{\f{\hbar m \omega}{2}} \left( a_{+} - a_{-} \right)
	\end{cases}
	\Rightarrow
	\begin{cases}
		X_{1}(t) & = \f{1}{2} \left( a_{+} + a_{-} \right)\\
		X_{2}(t) & = \f{i}{2} \left( a_{+} - a_{-} \right)
	\end{cases}
\end{align}		
\subsection*{Bài tập 2}
Ta có
\begin{align*}
	\begin{cases}
		X_{1}(t) & = \f{1}{2} \left( a_{+} + a_{-} \right)\\
		X_{2}(t) & = \f{i}{2} \left( a_{+} - a_{-} \right)
	\end{cases}
\end{align*}
nên 
\begin{align*}
	&\begin{cases}
		\ev{X_{1}} \propto \bra{n} a_{+} + a_{-} \ket{n} = 0\\
		\ev{X_{2}} \propto \bra{n} a_{+} - a_{-} \ket{n} = 0
	\end{cases} \\
	&\begin{cases}
		\ev{X_{1}^{2}} \propto \bra{n} a_{+}a_{+} + a_{-}a_{+} + a_{+}a_{-} + a_{-}a_{-} \ket{n} = \f{n}{2} + \f{1}{4}\\
		\ev{X_{2}^{2}} \propto \bra{n} a_{+}a_{+} - a_{+}a_{-} - a_{-}a_{+} + a_{-}a_{-} \ket{n} =
		\f{n}{2} + \f{1}{4} 
	\end{cases}
\end{align*}
và
\begin{align*}
	\Delta X_{1} &= \sqrt{\ev{X_{1}^{2}} - \ev{X_{1}}^{2}} = \sqrt{\f{n}{2} + \f{1}{4}}\\	
	\Delta X_{2} &= \sqrt{\ev{X_{2}^{2}} - \ev{X_{2}}^{2}} = \sqrt{\f{n}{2} + \f{1}{4}}	
\end{align*}
nên ta có 
\begin{align*}
	\Delta X_{1}^{\text{VAC}} &= \sqrt{\ev{X_{1}^{2}} - \ev{X_{1}}^{2}} = \sqrt{\f{n}{2} + \f{1}{4}}\;\at{n = 0}{} = \f{1}{2}\\	
	\Delta X_{2}^{\text{VAC}} &= \sqrt{\ev{X_{2}^{2}} - \ev{X_{2}}^{2}} = \sqrt{\f{n}{2} + \f{1}{4}} \;\at{n = 0}{} = \f{1}{2}
\end{align*}
\subsection*{Bài tập 3}
Xét trạng thái cohenrent $\ket{\alpha}$ với $\alpha = \abs{\alpha} e^{i \varphi}$. Ta viết lại trạng thái cohenrent dưới dạng $\propto \ket{n}$
%\begin{align*}
%	\ket{\alpha} = e^{ -\frac{\alpha^{2}}{2}} \sum_{n = 0}^{\infty} \f{\abs{\alpha}^{n} e^{i n \varphi}}{\sqrt{n !}} \ket{n}
%\end{align*}	
\begin{align*}
	a_{-}\ket{\alpha} = \abs{\alpha} e^{i \varphi}\ket{\alpha}
\end{align*}	
%Dẫn tới
%\begin{align*}
%	\ev{X_{1}} 
%	&= \bra{\alpha} X_{1} \ket{\alpha}\\
%	&= e^{ - ( \alpha^2 + \alpha^2 )/2 } \sum_{n,m = 0}^{\infty} \bra{m} \f{\abs{\alpha}^{n}\abs{\alpha}^{m} e^{i (n - m) \varphi }}{\sqrt{m!} \sqrt{n!}} (a_{+} + a_{-}) \ket{n}\\
%	&\propto \sum_{n,m = 0}^{\infty}  \f{\abs{\alpha}^{n}\abs{\alpha}^{m} e^{i (n - m) \varphi }}{\sqrt{m!} \sqrt{n!}} \bra{m} (a_{+} + a_{-}) \ket{n}\\
%	&\propto \sum_{n,m = 0}^{\infty}  \f{\abs{\alpha}^{n}\abs{\alpha}^{m} e^{i (n - m) \varphi }}{\sqrt{m!} \sqrt{n!}} \left(\bra{m} \underbrace{a_{+} \ket{n}}_{\sqrt{n+1} \ket{n + 1}} + \bra{m} \underbrace{a_{-} \ket{n}}_{\sqrt{n}\ket{n-1} }\right) \\
%	&\propto e^{- i \varphi} \sum_{n = 0} \f{\abs{\alpha}^{2n + 1}}{n!} + e^{i \varphi} \sum_{n = 0} \f{\abs{\alpha}^{2n - 1}}{(n-1)!}
%\end{align*}
%đặt n' = n - 1, ta có
%\begin{align*}
%	\ev{X_{1}} 
%	&= \propto  e^{- i \varphi} \sum_{n = 0} \f{\abs{\alpha}^{2n + 1}}{n!} + e^{i \varphi} \sum_{n = 0} \f{\abs{\alpha}^{2n - 1}}{n!}\\
%	&= \propto 2 \cos \varphi \sum_{n = 0} \f{\abs{\alpha}^{2n + 1} }{n!} \tag{4}
%\end{align*}
%đặt $\alpha^{2n} = \alpha $ nên (4) trở thành
%\begin{align*}
%	\ev{X_{1}} \propto 2 \abs{\alpha} \cos \varphi e^{\abs{\alpha}^2}
%\end{align*}
%Tương tự với $\ev{X_{2}}$
%\begin{align*}
%	\ev{X_{1}} \propto 2 \abs{\alpha} \cos \varphi e^{\abs{\alpha}^2}\\
%	\ev{X_{2}} \propto 2i \abs{\alpha} \sin \varphi e^{\abs{\alpha}^2}
%\end{align*}	
Dẫn tới
\begin{align*}
	\ev{X_{1}}
	&= \bra{\alpha} X_{1} \ket{\alpha}\\
	&\propto \bra{\alpha} a_{+} \ket{\alpha} + \bra{\alpha} a_{-} \ket{\alpha}
\end{align*}
ta có $a_{-} \ket{\alpha} = \abs{\alpha} e^{i \varphi}\ket{\alpha} \Leftrightarrow \bra{\alpha} a_{+} = \bra{\alpha}  e^{-i\varphi} \abs{\alpha} $, nên ta có
\begin{align*}
	\ev{X_{1}} 
	&\propto e^{- i \varphi} \abs{\alpha} \bra{\alpha} \ket{\alpha} + e^{i \varphi} \abs{\alpha} \bra{\alpha} \ket{\alpha}\\
	&\propto 2 \abs{\alpha} \cos \varphi
\end{align*}
tương tự
\begin{align*}
	\ev{X_{2}} 
	&\propto e^{- i \varphi} \abs{\alpha} \bra{\alpha} \ket{\alpha} + e^{i \varphi} \abs{\alpha} \bra{\alpha} \ket{\alpha}\\
	&\propto 2 \abs{\alpha} \sin \varphi
\end{align*}
\subsection*{Bài tập 4}
Ta viết lại toán tử $X_{1}^{2} + X_{2}^{2}$
\begin{align*}
	\begin{cases}
		X_{1}^{2} =\f{1}{4}  \left( a_{+} a_{+} + a_{-} a_{+} + a_{+} a_{-} + a_{-} a_{-}  \right)\\
		X_{2}^{2} =-\f{1}{4}  \left( a_{+} a_{+} - a_{-} a_{+} - a_{+} a_{-} + a_{-} a_{-}  \right)
	\end{cases}
\end{align*}
ta có
\begin{align*}
	X_{1}^{2} + X_{2}^{2} = \f{1}{2} \left(a_{-} a_{+} + a_{+} a_{-}\right)
\end{align*}	
nên dẫn ra được
\begin{align*}
	\left(X_{1}^{2} + X_{2}^{2}\right) \ket{n} 
	&= \f{1}{2} \left(a_{-} a_{+} + a_{+} a_{-}\right) \ket{n}\\
	&= \f{1}{2} a_{-} a_{+}  \ket{n} + \f{1}{2} a_{+} a_{-}  \ket{n} \\
	&= \f{1}{2} \sqrt{n+1}a_{-} \ket{n+1} + \f{1}{2} \sqrt{n} a_{+} \ket{n-1}\\
	&= \f{1}{2} (n + 1)\ket{n} + \f{n}{2} \ket{n}\\
	&= (n + \f{1}{2}) \ket{n}.
\end{align*}
nên ta có được ĐPCM.\\
Đánh giá số photon dựa trên $\Delta X_{1} \Delta X_{2}$.
\begin{align*}
	\left[ X_{1} , X_{2} \right] 
	&= X_{1} X_{2} - X_{2} X_{1}\\
	&= \f{i}{4} \left[ \left( a_{+} a_{+} + a_{-} a_{+} + a_{+} a_{-} + a_{-} a_{-} \right) + \left( a_{+} a_{+} - a_{-} a_{+} + a_{+} a_{-} - a_{-} a_{-} \right) \right]\\
	& = \f{i}{2} \left[ a_{-} , a_{+} \right] = \f{i}{2}
\end{align*}
nên ta có
\begin{align*}
	\Delta X_{1} \Delta X_{2} \geq 1 / 4 
\end{align*}
\subsection*{Bài tập 5}
Ta có
\begin{align*}
	&g^{(2)}(\tau) = \f{\ev{a_{3}^{+}(t) a_{4}^{+}(t + \tau) a_{4}^{-}(t + \tau)} a_{3}^{-}(t)}{\ev{a_{3}^{+}(t+ \tau) a_{3}^{-}(t+ \tau)} \ev{a_{4}^{+}(t + \tau) a_{4}^{-}(t+ \tau) }}\\
	\Rightarrow & g^{(2)}(0) = \f{\ev{a_{+} a_{+} a_{-} a_{-}}}{\ev{a_{+} a_{-}} \ev{a_{+} a_{-}}}
\end{align*}
Xét tử số
\begin{align*}
	\ev{a_{+} a_{+} a_{-} a_{-}} 
	&= \bra{\alpha} a_{+} a_{+} a_{-} a_{-} \ket{\alpha}\\
	&= \bra{\alpha} a_{+} ( a_{-} a_{+} - 1) a_{-} \ket{\alpha}\\
	&= \bra{\alpha} a_{+} a_{-} a_{+}  a_{-} \ket{\alpha} - \bra{\alpha} a_{+} a_{-} \ket{\alpha}\\
	&= \abs{\alpha}^{4} - \abs{\alpha}^{2}
\end{align*}	
Xét mẫu số
\begin{align*}
	\ev{a_{+} a_{-}} = \bra{\alpha} a_{+} a_{-} \ket{\alpha} = \abs{\alpha}^{2}
\end{align*}	
nên 
\begin{align*}
	g^{(2)}(0)
	&= \f{\abs{\alpha}^{4} - \abs{\alpha}^{2}}{\abs{\alpha}^{4}}\\
	&= 1 - \f{1}{\abs{\alpha}^{2}} = 1 ( \alpha \rightarrow \infty)
\end{align*}





































































































































































































































































































































































































































































































































































































































































































































































































































































































































































































































































































































































































































































































































































































































































































































\end{document}