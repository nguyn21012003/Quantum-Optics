\documentclass{report}
\usepackage[utf8]{vietnam}
\usepackage[utf8]{inputenc}
\usepackage{anyfontsize,fontsize}
\changefontsize[13pt]{13pt}	
\usepackage{commath}
\usepackage{parskip}
\usepackage{xcolor}
\usepackage{amssymb}
\usepackage{slashed,cancel}
\usepackage{indentfirst}
\usepackage{pdfpages}
\usepackage{graphicx}
\usepackage{nccmath}
\usepackage{mathtools}
\usepackage{amsfonts}
\usepackage{amsmath,systeme}
\usepackage[thinc]{esdiff}
\usepackage{hyperref}
\usepackage{bm,physics,nicematrix}
\usepackage{fancyhdr}
%footnote
\pagestyle{fancy}
\renewcommand{\headrulewidth}{0pt}%
\fancyhf{}%
\fancyfoot[L]{Vật lý Lý thuyết}%
\fancyfoot[C]{\hspace{4cm} \thepage}%


\usepackage{geometry}
\geometry{
	a4paper,
	total={170mm,257mm},
	left=20mm,
	top=20mm,
}


\newcommand{\image}[1]{
	\begin{center}
		\includegraphics[width=0.5\textwidth]{pic/#1}
	\end{center}
}
\renewcommand{\l}{\ell}
\newcommand{\dps}{\displaystyle}

\newcommand{\f}[2]{\dfrac{#1}{#2}}
\newcommand{\at}[2]{\bigg\rvert_{#1}^{#2} }


\renewcommand{\baselinestretch}{2.0}


\title{\Huge{Quantum Optics}}

\hypersetup{
	colorlinks=true,
	linkcolor=red,
	filecolor=magenta,      
	urlcolor=cyan,
	pdftitle={QM3},
	pdfpagemode=FullScreen,
}

\urlstyle{same}

\begin{document}
\setlength{\parindent}{20pt}
\newpage
\author{TRẦN KHÔI NGUYÊN \\ VẬT LÝ LÝ THUYẾT}
\maketitle

\subsection*{Bài tập 1}
\begin{align*}
	X_{1} (t) &= \left(\f{\omega}{2 \hbar}\right)^{1/2} q(t) \\
	X_{2} (t) &= \left(\f{1}{2 \hbar \omega}\right)^{1/2} p(t)
\end{align*}
ta biểu diễn toạ độ chính tắc và động lượng suy rộng dưới dạng $x$ và $p_{x}$
\begin{align}
	\begin{cases}
		X_{1} (t) &= \left(\f{\omega}{2 \hbar}\right)^{1/2} \sqrt{m} x(t) \\
		X_{2} (t) &= \left(\f{1}{2 \hbar \omega}\right)^{1/2} \f{1}{\sqrt{m}} p_{x}(t)
	\end{cases}
\end{align}	
ta có 
\begin{align*}
	a_{\pm} \equiv \f{1}{\sqrt{2 \hbar m \omega}} \left( \mp i p + m \omega x \right),
\end{align*}
và
\begin{align}
	\begin{cases}
		x = \sqrt{\f{\hbar}{2 m \omega}} \left( a_{+} + a_{-} \right)\\
		p = i \sqrt{\f{\hbar m \omega}{2}} \left( a_{+} - a_{-} \right)
	\end{cases}
\end{align}	
Thay (2) vào (1), ta được
\begin{align}
	\begin{cases}
		X_{1} (t) &= \left(\f{\omega}{2 \hbar}\right)^{1/2} \sqrt{m} \sqrt{\f{\hbar}{2 m \omega}} \left( a_{+} + a_{-} \right) \\
		X_{2} (t) &= \left(\f{1}{2 \hbar \omega}\right)^{1/2} \f{1}{\sqrt{m}}  i \sqrt{\f{\hbar m \omega}{2}} \left( a_{+} - a_{-} \right)
	\end{cases}
	\Rightarrow
	\begin{cases}
		X_{1}(t) & = \f{1}{2} \left( a_{+} + a_{-} \right)\\
		X_{2}(t) & = \f{i}{2} \left( a_{+} - a_{-} \right)
	\end{cases}
\end{align}		
\subsection*{Bài tập 2}
Ta có
\begin{align*}
	\begin{cases}
		X_{1}(t) & = \f{1}{2} \left( a_{+} + a_{-} \right)\\
		X_{2}(t) & = \f{i}{2} \left( a_{+} - a_{-} \right)
	\end{cases}
\end{align*}
nên 
\begin{align*}
	&\begin{cases}
		\ev{X_{1}} \propto \bra{n} a_{+} + a_{-} \ket{n} = 0\\
		\ev{X_{2}} \propto \bra{n} a_{+} - a_{-} \ket{n} = 0
	\end{cases} \\
	&\begin{cases}
		\ev{X_{1}^{2}} \propto \bra{n} a_{+}a_{+} + a_{-}a_{+} + a_{+}a_{-} + a_{-}a_{-} \ket{n} = \f{n}{2} + \f{1}{4}\\
		\ev{X_{2}^{2}} \propto \bra{n} a_{+}a_{+} - a_{+}a_{-} - a_{-}a_{+} + a_{-}a_{-} \ket{n} =
		\f{n}{2} + \f{1}{4} 
	\end{cases}
\end{align*}
và
\begin{align*}
	\Delta X_{1} &= \sqrt{\ev{X_{1}^{2}} - \ev{X_{1}}^{2}} = \sqrt{\f{n}{2} + \f{1}{4}}\\	
	\Delta X_{2} &= \sqrt{\ev{X_{2}^{2}} - \ev{X_{2}}^{2}} = \sqrt{\f{n}{2} + \f{1}{4}}	
\end{align*}
nên ta có 
\begin{align*}
	\Delta X_{1}^{\text{VAC}} &= \sqrt{\ev{X_{1}^{2}} - \ev{X_{1}}^{2}} = \sqrt{\f{n}{2} + \f{1}{4}}\;\at{n = 0}{} = \f{1}{2}\\	
	\Delta X_{2}^{\text{VAC}} &= \sqrt{\ev{X_{2}^{2}} - \ev{X_{2}}^{2}} = \sqrt{\f{n}{2} + \f{1}{4}} \;\at{n = 0}{} = \f{1}{2}
\end{align*}
\subsection*{Bài tập 3}
Xét trạng thái cohenrent $\ket{\alpha}$ với $\alpha = \abs{\alpha} e^{i \varphi}$. Ta viết lại trạng thái cohenrent dưới dạng $\propto \ket{n}$
%\begin{align*}
%	\ket{\alpha} = e^{ -\frac{\alpha^{2}}{2}} \sum_{n = 0}^{\infty} \f{\abs{\alpha}^{n} e^{i n \varphi}}{\sqrt{n !}} \ket{n}
%\end{align*}	
\begin{align*}
	a_{-}\ket{\alpha} = \abs{\alpha} e^{i \varphi}\ket{\alpha}
\end{align*}	
%Dẫn tới
%\begin{align*}
%	\ev{X_{1}} 
%	&= \bra{\alpha} X_{1} \ket{\alpha}\\
%	&= e^{ - ( \alpha^2 + \alpha^2 )/2 } \sum_{n,m = 0}^{\infty} \bra{m} \f{\abs{\alpha}^{n}\abs{\alpha}^{m} e^{i (n - m) \varphi }}{\sqrt{m!} \sqrt{n!}} (a_{+} + a_{-}) \ket{n}\\
%	&\propto \sum_{n,m = 0}^{\infty}  \f{\abs{\alpha}^{n}\abs{\alpha}^{m} e^{i (n - m) \varphi }}{\sqrt{m!} \sqrt{n!}} \bra{m} (a_{+} + a_{-}) \ket{n}\\
%	&\propto \sum_{n,m = 0}^{\infty}  \f{\abs{\alpha}^{n}\abs{\alpha}^{m} e^{i (n - m) \varphi }}{\sqrt{m!} \sqrt{n!}} \left(\bra{m} \underbrace{a_{+} \ket{n}}_{\sqrt{n+1} \ket{n + 1}} + \bra{m} \underbrace{a_{-} \ket{n}}_{\sqrt{n}\ket{n-1} }\right) \\
%	&\propto e^{- i \varphi} \sum_{n = 0} \f{\abs{\alpha}^{2n + 1}}{n!} + e^{i \varphi} \sum_{n = 0} \f{\abs{\alpha}^{2n - 1}}{(n-1)!}
%\end{align*}
%đặt n' = n - 1, ta có
%\begin{align*}
%	\ev{X_{1}} 
%	&= \propto  e^{- i \varphi} \sum_{n = 0} \f{\abs{\alpha}^{2n + 1}}{n!} + e^{i \varphi} \sum_{n = 0} \f{\abs{\alpha}^{2n - 1}}{n!}\\
%	&= \propto 2 \cos \varphi \sum_{n = 0} \f{\abs{\alpha}^{2n + 1} }{n!} \tag{4}
%\end{align*}
%đặt $\alpha^{2n} = \alpha $ nên (4) trở thành
%\begin{align*}
%	\ev{X_{1}} \propto 2 \abs{\alpha} \cos \varphi e^{\abs{\alpha}^2}
%\end{align*}
%Tương tự với $\ev{X_{2}}$
%\begin{align*}
%	\ev{X_{1}} \propto 2 \abs{\alpha} \cos \varphi e^{\abs{\alpha}^2}\\
%	\ev{X_{2}} \propto 2i \abs{\alpha} \sin \varphi e^{\abs{\alpha}^2}
%\end{align*}	
Dẫn tới
\begin{align*}
	\ev{X_{1}}
	&= \bra{\alpha} X_{1} \ket{\alpha}\\
	&\propto \bra{\alpha} a_{+} \ket{\alpha} + \bra{\alpha} a_{-} \ket{\alpha}
\end{align*}
ta có $a_{-} \ket{\alpha} = \abs{\alpha} e^{i \varphi}\ket{\alpha} \Leftrightarrow \bra{\alpha} a_{+} = \bra{\alpha}  e^{-i\varphi} \abs{\alpha} $, nên ta có
\begin{align*}
	\ev{X_{1}} 
	&\propto e^{- i \varphi} \abs{\alpha} \bra{\alpha} \ket{\alpha} + e^{i \varphi} \abs{\alpha} \bra{\alpha} \ket{\alpha}\\
	&\propto 2 \abs{\alpha} \cos \varphi
\end{align*}
tương tự
\begin{align*}
	\ev{X_{2}} 
	&\propto e^{- i \varphi} \abs{\alpha} \bra{\alpha} \ket{\alpha} + e^{i \varphi} \abs{\alpha} \bra{\alpha} \ket{\alpha}\\
	&\propto 2 \abs{\alpha} \sin \varphi
\end{align*}
\subsection*{Bài tập 4}
Ta viết lại toán tử $X_{1}^{2} + X_{2}^{2}$
\begin{align*}
	\begin{cases}
		X_{1}^{2} =\f{1}{4}  \left( a_{+} a_{+} + a_{-} a_{+} + a_{+} a_{-} + a_{-} a_{-}  \right)\\
		X_{2}^{2} =-\f{1}{4}  \left( a_{+} a_{+} - a_{-} a_{+} - a_{+} a_{-} + a_{-} a_{-}  \right)
	\end{cases}
\end{align*}
ta có
\begin{align*}
	X_{1}^{2} + X_{2}^{2} = \f{1}{2} \left(a_{-} a_{+} + a_{+} a_{-}\right)
\end{align*}	
nên dẫn ra được
\begin{align*}
	\left(X_{1}^{2} + X_{2}^{2}\right) \ket{n} 
	&= \f{1}{2} \left(a_{-} a_{+} + a_{+} a_{-}\right) \ket{n}\\
	&= \f{1}{2} a_{-} a_{+}  \ket{n} + \f{1}{2} a_{+} a_{-}  \ket{n} \\
	&= \f{1}{2} \sqrt{n+1}a_{-} \ket{n+1} + \f{1}{2} \sqrt{n} a_{+} \ket{n-1}\\
	&= \f{1}{2} (n + 1)\ket{n} + \f{n}{2} \ket{n}\\
	&= (n + \f{1}{2}) \ket{n}.
\end{align*}
nên ta có được ĐPCM.\\
Đánh giá số photon dựa trên $\Delta X_{1} \Delta X_{2}$.
\begin{align*}
	\left[ X_{1} , X_{2} \right] 
	&= X_{1} X_{2} - X_{2} X_{1}\\
	&= \f{i}{4} \left[ \left( a_{+} a_{+} + a_{-} a_{+} + a_{+} a_{-} + a_{-} a_{-} \right) + \left( a_{+} a_{+} - a_{-} a_{+} + a_{+} a_{-} - a_{-} a_{-} \right) \right]\\
	& = \f{i}{2} \left[ a_{-} , a_{+} \right] = \f{i}{2}
\end{align*}
nên ta có
\begin{align*}
	\Delta X_{1} \Delta X_{2} \geq 1 / 4 
\end{align*}
\subsection*{Bài tập 5}
Ta có
\begin{align*}
	&g^{(2)}(\tau) = \f{\ev{a_{3}^{+}(t) a_{4}^{+}(t + \tau) a_{4}^{-}(t + \tau)} a_{3}^{-}(t)}{\ev{a_{3}^{+}(t+ \tau) a_{3}^{-}(t+ \tau)} \ev{a_{4}^{+}(t + \tau) a_{4}^{-}(t+ \tau) }}\\
	\Rightarrow & g^{(2)}(0) = \f{\ev{a_{+} a_{+} a_{-} a_{-}}}{\ev{a_{+} a_{-}} \ev{a_{+} a_{-}}}
\end{align*}
Xét tử số
\begin{align*}
	\ev{a_{+} a_{+} a_{-} a_{-}} 
	&= \bra{\alpha} a_{+} a_{+} a_{-} a_{-} \ket{\alpha}\\
	&= \bra{\alpha} a_{+} ( a_{-} a_{+} - 1) a_{-} \ket{\alpha}\\
	&= \bra{\alpha} a_{+} a_{-} a_{+}  a_{-} \ket{\alpha} - \bra{\alpha} a_{+} a_{-} \ket{\alpha}\\
	&= \abs{\alpha}^{4} - \abs{\alpha}^{2}
\end{align*}	
Xét mẫu số
\begin{align*}
	\ev{a_{+} a_{-}} = \bra{\alpha} a_{+} a_{-} \ket{\alpha} = \abs{\alpha}^{2}
\end{align*}	
nên 
\begin{align*}
	g^{(2)}(0)
	&= \f{\abs{\alpha}^{4} - \abs{\alpha}^{2}}{\abs{\alpha}^{4}}\\
	&= 1 - \f{1}{\abs{\alpha}^{2}} = 1 ( \alpha \rightarrow \infty)
\end{align*}	
	
	
	
	
	
	
	
	
	
	
	
	
	
	
	
\end{document}