\documentclass{article}
\usepackage[utf8]{vietnam}
\usepackage[utf8]{inputenc}
\usepackage{anyfontsize,fontsize}
\changefontsize[13pt]{13pt}
\usepackage{commath}
\usepackage{parskip}
\usepackage{xcolor}
\usepackage{amssymb}
\usepackage{slashed,cancel}
\usepackage{indentfirst}
\usepackage{pdfpages}
\usepackage{graphicx}
\usepackage{nccmath}
\usepackage{mathtools}
\usepackage{amsfonts}
\usepackage{amsmath,systeme}
\usepackage{esdiff}
\usepackage{hyperref}
\usepackage{bm,physics}
\usepackage{fancyhdr}
%footnote
\pagestyle{fancy}
\renewcommand{\headrulewidth}{0pt}%
\fancyhf{}%
\fancyfoot[L]{Vật lý Lý thuyết}%
\fancyfoot[C]{\hspace{4cm} \thepage}%
\usepackage[d]{esvect}


\usepackage{geometry}
\geometry{
	a4paper,
	total={170mm,257mm},
	left=20mm,
	top=20mm,
}


\newcommand{\image}[1]{
	\begin{center}
		\includegraphics[width=0.5\textwidth]{pic/#1}
	\end{center}
}
\renewcommand{\l}{\ell}
\newcommand{\dps}{\displaystyle}

\newcommand{\f}[2]{\dfrac{#1}{#2}}
\newcommand{\at}[2]{\bigg\rvert_{#1}^{#2} }


\renewcommand{\baselinestretch}{2.0}


\title{\Huge{BÀI TẬP VỀ NHÀ 2}}

\hypersetup{
	colorlinks=true,
	linkcolor=red,
	filecolor=magenta,      
	urlcolor=cyan,
	pdftitle={QO},
	pdfpagemode=FullScreen,
}

\urlstyle{same}

\begin{document}
\setlength{\parindent}{20pt}
\newpage
\author{TRẦN KHÔI NGUYÊN \\ VẬT LÝ LÝ THUYẾT}
\maketitle
%\subsection*{``Photon statistics'' nào cho loại ánh sáng nào?}
%\begin{enumerate}
%	\item[(a)] \textbf{Coherent light}\\
%	Xét trường ánh sáng ``cohenrent'' có biên đọ $E_0$, tần số $\omega$, và pha $\varphi$, với trường điện có phương trình là
%	\begin{align*}
%		\mathcal{E}(x,t) = \mathcal{E}_0 \sin(kx - \omega t + \varphi).
%	\end{align*}
%	Cường độ $I \propto $ bình phương biên độ và là hằng số nếu $E_0$ và $\varphi$ không phụ thuộc vào thời gian(trường hợp coherent hoàn toàn) $\rightarrow$ không có thăng giáng cường độ, và thông lượng photon là không đổi theo thời gian. Trong thời gian cực ngắn, ta vẫn luôn có sự thăng giáng thống kê do bản chất rời rạc của photon.\\
%	Xét chùm tia có công suất $P$ không đổi. Số photon trung bình trong đoạn dài $L$(tức là trong khoảng thời gian $T=L/c$) của chùm tia này được cho bởi công thức
%	\begin{align*}
%		\overline{n} = \Phi \f{L}{c},
%	\end{align*}
%	với $\Phi \equiv \f{P}{\hbar \omega}$(photon / s) là thông lượng. Ta giả định rằng $L$ phải đủ lớn, để trong mỗi đoạn chia $dL$ ta nhận được số nguyên lần photon. $N$ cũng phải là đủ lớn
%\end{enumerate}

\subsection*{Chứng minh rằng $(\Delta n)^2 = \dps\sum_{n} (n - \overline{n}) P_{\omega}(n) = \overline{n} + \overline{n}^2$}
Giải: \\
Đặt $x \equiv \exp(-\f{\hbar \omega}{k_B T})$, mà từ công thức (5.20) Mark Fox, ta biết $P(\omega)$ có dạng
\begin{align*}
	P_{\omega}(n)
	 & = \f{\exp(-E_{n} / k_B T)}{\sum_{n = 0}^{\infty}(\exp(-E_{n} / k_B T)}                   \\
	 & = \f{\exp(-n \hbar \omega / k_B T)}{\sum_{n = 0}^{\infty}(\exp(n \hbar \omega / k_B T)}, \\
	 & = \f{x^{n}}{\sum_{n = 0}^{\infty} x^{n}}. \tag{1}
\end{align*}
Ta xét chuỗi hình học
\begin{align*}
	\sum_{n = 0}^{\infty} x^{n} = \f{1 - x^n}{1 - x} \tag{2},
\end{align*}
khi $x<1$, thì (2) trở thành
\begin{align*}
	\sum_{n = 0}^{\infty} x^{n} = \f{1}{1 - x}, \tag{3}
\end{align*}
dẫn đến ta có thể viết lại $P_{\omega}(n)$
\begin{align*}
	P_{\omega}(n)
	 & = x^{n} (1-x)                                                                               \\
	 & \equiv \left( 1 - \exp(-\hbar \omega / k_B T) \right) \exp(-n\hbar \omega / k_B T). \tag{4}
\end{align*}
Số photon trung bình được cho bởi
\begin{align*}
	\overline{n}
	 & = \sum_{n = 0}^{\infty} n P_{\omega}(n)                                                                                       \\
	 & = \sum_{n = 0}^{\infty} n x^{n} (1-x)                                                                                         \\
	 & = (1 - x) x \f{d}{dx} \left( \sum_{n = 0}^{\infty} x^{n} \right) (\text{Tại vì đạo hàm thì số bậc của $x$ là $n - 1$})        \\
	 & = (1 - x) x \f{d}{dx} \left(\f{1}{1 - x}\right)                                                                               \\
	 & = (1 - x) x \f{1}{(1 - x)^2}                                                                                                  \\
	 & = \f{x}{1 - x}                                                                                                                \\
	 & = \f{\exp(-\f{\hbar \omega}{k_B T})}{1 - \exp(-\f{\hbar \omega}{k_B T})} = \f{1}{\exp(\f{\hbar \omega}{k_B T}) - 1} . \tag{5}
\end{align*}
Phương trình (5) dẫn đến được
\begin{align*}
	x = \f{\overline{n}}{\overline{n} + 1},
\end{align*}
ta thay vô phương trình $P_{\omega}(n) = x^{n} (1-x)$, và được
\begin{align*}
	P_{\omega}(n)
	 & = \left(1 - \f{\overline{n}}{\overline{n} + 1}\right) \left(\f{\overline{n}}{\overline{n} + 1}\right)^{n} \\
	 & = \left(\f{1}{\overline{n} + 1}\right)\left(\f{\overline{n}}{\overline{n} + 1}\right)^{n} .
\end{align*}
Phương sai Var($n$) $\equiv \Delta n$
\begin{align*}
	(\Delta n)^2
	 & = \sum_{n = 0}^{\infty}(n - \overline{n})^2 P_{\omega}(n)                                                                                                                                                                  \\
	 & = \sum_{n = 0}^{\infty}(n^2 - 2n\overline{n} + \overline{n}^2) x^{n} (1-x)                                                                                                                                                 \\
	 & = \sum_{n = 0}^{\infty}n^2 x^{n} (1-x) - \sum_{n = 0}^{\infty}2n\overline{n} x^{n} (1-x) + \sum_{n = 0}^{\infty}\overline{n}^2 x^{n} (1-x)                                                                                 \\
	 & = (1-x)\sum_{n = 0}^{\infty}n^2 x^{n} - 2\overline{n}(1-x)\underbracket{\sum_{n = 0}^{\infty}n x^{n}}_{\frac{x}{(1 - x)^2}}  +\, \overline{n}^2(1-x) \underbracket{\sum_{n = 0}^{\infty} x^{n}}_{\frac{1}{1 - x}}. \tag{6}
\end{align*}
Xét số hạng đầu tiên $(1-x)\sum_{n = 0}^{\infty}n^2 x^{n}$, ta có
\begin{align*}
	\f{d}{dx} \left( \sum_{n = 0}^{\infty} x^{n} \right)
	                                                         & = \sum_{n = 0}^{\infty} n x^{n-1}                          \\
	\Rightarrow \f{d^2}{dx^2} \left( \sum_{n = 0}^{\infty} x^{n} \right)
	                                                         & = \f{d}{dx} \left( \sum_{n = 0}^{\infty} n x^{n-1} \right) \\
	\f{d^2}{dx^2} \left( \sum_{n = 0}^{\infty} x^{n} \right) & = \sum_{n = 0}^{\infty} n(n-1) x^{n-2}                     \\
	\f{d^2}{dx^2} \left( \sum_{n = 0}^{\infty} x^{n} \right) & = \sum_{n = 0}^{\infty} (n^2-n) x^{n-2}. \tag{7}
\end{align*}
Nhân 2 vế của (6) cho $x^2$ ta được
\begin{align*}
	x^2 \f{d^2}{dx^2} \left( \sum_{n = 0}^{\infty} x^{n} \right)
	 & = \sum_{n = 0}^{\infty} (n^2-n) x^{n}
	= \sum_{n = 0}^{\infty} n^2 x^{n} - \sum_{n = 0}^{\infty} nx^{n}                                                                           \\
	\Rightarrow \sum_{n = 0}^{\infty} n^2 x^{n}
	 & = x^2 \f{d^2}{dx^2} \left( \sum_{n = 0}^{\infty} x^{n} \right) + \sum_{n = 0}^{\infty} nx^{n}                                           \\
	 & = x^2 \f{d^2}{dx^2} \left( \sum_{n = 0}^{\infty} x^{n} \right) + x \f{d}{dx} \left( \sum_{n = 0}^{\infty} x^{n} \right) .(DPCM) \tag{8}
\end{align*}
Thay vô (6) ta được
\begin{align*}
	(\Delta n)^2
	 & = (1-x) \left[x^2 \f{d^2}{dx^2} \left( \sum_{n = 0}^{\infty} x^{n} \right) + x \f{d}{dx} \left( \sum_{n = 0}^{\infty} x^{n} \right)\right] - \f{2\overline{n}(1-x)x}{(1 - x)^2} + \overline{n}^2 \\
	 & = (1-x) \left[x^2 \f{d^2}{dx^2} \left( \f{1}{1 - x} \right) + x \f{d}{dx} \left( \f{1}{1 - x} \right)\right] - \f{2\overline{n}(1-x)x}{(1 - x)^2} + \overline{n}^2                               \\
	 & = (1-x) \left[\f{2x^2}{(1-x)^3} + \f{x}{(1 - x)^2}\right] - \f{2\overline{n}(1-x)x}{(1 - x)^2} + \overline{n}^2                                                                                  \\
	 & = \left[2 \overline{n}^2 + \overline{n}\right] - 2 \overline{n}^2 + \overline{n}^2                                                                                                               \\
	 & = \overline{n} + \overline{n}^2 (DPCM).\tag{9}
\end{align*}

\subsection*{Chứng minh rằng $\langle \Delta E^2 \rangle = k_B T^2 \f{\partial \langle E \rangle}{T}$}

\subsection*{Chứng minh rằng $D(\omega) = \frac{\omega^2}{\pi^2 c^3}$}
Giải:\\
Mỗi trạng thái $\mathbf{k}$ trong mạng có 1 cặp trạng thái 1 hạt.
\begin{align*}
	\begin{cases}
		\text{2 trạng thái chiếm thể tích :} \; \left(\f{2\pi}{L}\right)^3 \\
		\text{? số trạng thái chiếm thể tích :} \; d\mathbf{k}             \\
	\end{cases}
\end{align*}
số trạng thái trong $d\mathbf{k}$ là
\begin{align*}
	D(k)dk = 2  \left(\f{L}{2\pi}\right)^3 d\mathbf{k} = \f{2V}{(2\pi)^3} 4\pi k^2 dk = \f{V}{\pi^2}k^2,
\end{align*}
mà
\begin{align*}
	D(\omega) D\omega     & = D(k) dk                 \\
	\Rightarrow D(\omega) & = \f{D(k)}{d\omega / dk}.
\end{align*}
Đặt $\omega = ck$ ta có được
\begin{align*}
	g(\omega) = \f{\omega^2}{\pi^2 c^3} (DPCM).
\end{align*}

\subsection*{Bài tập trong Mark Fox}
\subsubsection*{5.1}
Ta có bước sóng $\lambda = 633$ nm, với công suất là $P = 0.01$ pW, với hiệu suất lượng tử là $\eta = 30\% $ với khoảng thời gian là $T = 10$ ms
\begin{enumerate}
	\item[(a)] Tốc độ đếm $\mathcal{R} = \f{\eta P}{\hbar \omega} = 0.3 \f{0.01 \times 10^{-12}}{\hbar \frac{2\pi c}{\lambda}} = 9553.174$ count/s.
	\item[(b)] Tốc độ đếm trung bình $N(T)= \eta \f{PT}{\hbar \omega} = 0.3 \times \mathcal{R}\times 0.01 =$
	\item[(c)] Ta đã giả sử rằng photon detected count có thống kê Poissonian, và đã biết là $\overline{n} = N$, do đó $\Delta n = \sqrt{\overline{n}}$.
\end{enumerate}
\subsubsection*{5.4}
Tính số photon trung bình cho từng mode tại nguồn là đèn Tungsten có bước sóng $\lambda =$ 500 nm, tại $T = $ 2000 K là nhiệt độ cần để có được $\overline{n} = 1$ tại bước sóng $\lambda$. Nhiệt độ tại bước sóng là $\lambda^{'} = 10 \mu$m. Từ công thức Plank
\begin{align*}
	\overline{n} = \f{1}{\exp(\frac{\hbar \omega}{k_{B} T}) - 1} = \frac{1}{\exp(\frac{hc}{\lambda k_{B} T}) - 1} = \f{1}{\exp(\frac{hc}{500\times 10^{-9} 1.38 \times 10^{-23} 2000}) - 1} \approx 5.548 \times 10^{-7}. %\text{photon}
\end{align*}
Để $\overline{n} = 1$ thì $T$ là bao nhiêu?
\begin{align*}
	\overline{n} = 1 = \f{1}{\exp(\frac{hc}{500\times 10^{-9} 1.38 \times 10^{-23} T}) - 1} \Rightarrow T = 41562.6 \text{K}.
\end{align*}
Với $\lambda = \lambda'$ và để $\overline{n} = 1$ thì $T$ là bao nhiêu?
\begin{align*}
	\overline{n} = 1 = \f{1}{\exp(\frac{hc}{10 \times 10^{-6} 1.38 \times 10^{-23} T}) - 1} \Rightarrow T = 2078.13 \text{K}.
\end{align*}
\subsubsection*{5.5}
\subsubsection*{5.8}
Một chùm tia với thông lượng photon là $\Phi = 1000$ photon s$^{-1}$ tới detector với hiệu suất lượng tử là $\eta = 20 \%$. Nếu khoảng thời gian được xét là $T = 10$s, tính trung bình và độ lệch chuẩn của $N(t)$ trong các trường hợp sau:
\begin{enumerate}
	\item[(a)] Ánh sáng có thống kê Poissonian
	\item[(b)] Ánh sáng có Super-Poissonian với độ lệch chuẩn $\Delta n = 2\times \Delta n_{\text{Poissonian}}$
	\item[(c)] Ánh sáng là trạng thái số photon
\end{enumerate}
Giải:
Số photon trung bình va chạm vào detector:
\begin{align*}
	\overline{n} = \Phi T = 10^4.
\end{align*}
Tốc độ đếm trung bình
\begin{align*}
	\overline{N} = \eta \overline{n} = 2\times 10^3.
\end{align*}
\begin{enumerate}
	\item[(a)] Ánh sáng có thống kê Photon, Average photon count number $N(t)$ là
	      \begin{align*}
		      \overline{N}(t) \equiv \overline{n}= \eta \Phi T = 0.2 \times 1000 \times 10 = 2000 \; \text{photon s}^{-1}.
	      \end{align*}
	      Độ lệch chuẩn $(\Delta \overline{N})$
	      \begin{align*}
		      (\Delta \overline{N})^2 = \overline{n} = \overline{N} \Rightarrow \Delta \overline{N} = \sqrt{\overline{N}} = \sqrt{2000}.
	      \end{align*}
	\item[(b)] Ánh sáng có Super-Poissonian\\
	      Số photon trung bình (Average/mean photon count number) là $\overline{N}(t) = 2000$. Ta có
	      \begin{align*}
		      (\Delta \overline{N})^2
		       & = \eta^2 (\Delta n)^2 + \eta ( 1 - \eta ) \overline{n} \tag{5.56 M.Fox}                                                                     \\
		       & =  \eta^2 \times (2 \times \Delta n_{\text{Poissonian}})^2 + \eta ( 1 - \eta ) \overline{n}                                                 \\
		       & = 0.2^2 \times 4 \times \overline{n} + 0.2 \times (1 - 0.2) \times \overline{n} = 3200 \Rightarrow \Delta N = \sqrt{3200}. \tag{5.15 M.Fox}
	      \end{align*}
	      trong đó $\Delta N$ là phương sai của photoncount number,
	\item[(c)]
	      Độ lệch chuẩn $(\Delta \overline{N})$
	      \begin{align*}
		      (\Delta \overline{N})^2 = \eta(1-\eta) \overline{n} = \sqrt{1600}.
	      \end{align*}
\end{enumerate}

\subsubsection*{5.9}
Cường độ dòng photon ($i$) là
\begin{align*}
	i = \eta e \f{P}{\hbar \omega} = 0.9 \abs{1.6\times 10^{-19}} \f{10 \times 10^{-3}}{\hbar c / \lambda} = 4.584 \text{A},
\end{align*}
noise power đi qua 1 đơn vị băng thông là
\begin{align*}
	P_{\text{noise}} = 2e \times 50 \times 4.584  = 458.4e \; \text{W Hz}^{-1}.
\end{align*}
\subsubsection*{5.12}
Xét dòng điện đi qua điện trở $R$, tại nhiệt độ $T$ K trong mạch điện trở. Sự thăng giáng của dòng điện bên trong giải tần số $\Delta f$ được cho bởi hệ thức Johson noise:
\begin{align*}
	\langle (\Delta i)^2 \rangle = 4 k_{B} T \f{\Delta f}{R} .
\end{align*}
Chứng minh rằng Johson shot noise cho kết quả nhỏ hơn shot noise với cùng một giá trị trung bình dòng điện được cho bởi điện thế đi qua điện trở là lớn hơn $\f{2k_B T}{e}$, và tính điện thế này với $T = 300$K.\\
Giải:\\
Ta có phương sai cường độ dòng điện $\Delta i$ cho shot noise là
\begin{align*}
	(\Delta i)_{\text{shot noise}}^2
	 & = 2e \Delta f \langle i \rangle                                 \\
	 & = 2e \Delta f \f{V}{R}                                          \\
	 & \equiv \langle (\Delta i)^2 \rangle_{\text{shot noise}} \tag{1}
\end{align*}
Hệ thức Johnson cho
\begin{align*}
	\langle (\Delta i)^2 \rangle_{\text{Johnson}} = 4 k_{B} T \f{\Delta f}{R}. \tag{2}
\end{align*}
Để (2) < (1) thì
\begin{align*}
	4 k_{B} T \f{\Delta f}{R} & < 2e \Delta f \f{V}{R}           \\
	\Rightarrow V             & > \f{2k_B T}{e}  \; \text{ĐPCM}.
\end{align*}
Điện thế này tại $T = 300$K
\begin{align*}
	V > \f{2k_B T}{e} = \f{2 \times 1.38 \times 10^{-23} \times 300}{1.6 \times 10^{-19}} = 0.05175 \text{V}.
\end{align*}

\subsubsection*{5.13}
Hiệu suất lượng tử của một bóng đèn LED được định nghĩa là tỉ số giữa số photon phát xạ so với số electron chạy qua máy dò. Một đèn LED phát ra ánh sáng ở tần số 800 nm, được định hướng bởi nguồn là một ắc quy có điện thế là 9V thông qua điện trở $R = 1000 \, \Omega$. Bóng đèn LED có hiệu suất lượng tử $\eta_{\text{LED}} = 40\%$, và có $\eta_{1} = \eta_{\text{optic}} = 80\%$ photon phát xạ được tập trung vào một máy dò photodiode với hiệu suất lượng tử là $\eta_{2} =\eta_{\text{Detect}} = 90\%$.
\begin{enumerate}
	\item[(a)] Năng lượng photon là
	      \begin{align*}
		      E_{\text{photon}} = \f{\hbar 2 \pi c}{\lambda} = 1.55eV.
	      \end{align*}
	      Biết rằng điện thế lên đèn LED là xấp xỉ với năng lượng photon, tính bằng eV trong điều kiện hoạt động bình thường, nên ta có:
	      \begin{align*}
		      U_{\text{LED}} = 1.55 V.
	      \end{align*}
	      Điện thế đi qua điện trở là:
	      \begin{align*}
		      U_{\text{nguồn}} = U_{\text{LED}} + U_{R} = 9 V \Rightarrow U_{R} = 9 - 1.55 = 7.45 \text{V}.
	      \end{align*}
	      Dòng điện đi qua $U_{R}$
	      \begin{align*}
		      I_{R} = \f{U_{R}}{R} = 7.45 \text{mA}.
	      \end{align*}
	      \image{circuit.png}
	      Do không có sự thăng giáng nên trung bình dòng điện sẽ bằng dòng điện đi qua $U_{R}$
	      \begin{align*}
		      i_{1} = I_{R} = 7.45 \text{mA}.
	      \end{align*}
	\item[(b)] Ta có $T = 293 K$\\
	      Hệ số $F_{\text{Fano}}$ được cho bởi
	      \begin{align*}
		      F_{\text{Fano}}
		       & = \f{\text{measured noise}}{\text{shot noise limit}}                  \\
		       & = \f{(\Delta i)_{\text{Johnson}}^2}{(\Delta i)_{\text{shot noise}}^2} \\
		       & = \f{4 k_B T \Delta f}{2e R \Delta f \langle i \rangle}               \\
		       & = 6.77 \times 10^{-3}
	      \end{align*}
	\item[(c)] Dòng photon trung bình được tính bởi
	      \begin{align*}
		      \langle i \rangle_{\text{photon}} = \eta e \Phi \tag{5.58 \text{M.Fox}}
	      \end{align*}
	      trong đó $\eta$ là hiệu suất lượng tử của photodiode detector, và $\Phi$ là thông lượng photon đi tới PD\\
	      \begin{align*}
		      \Phi = \f{P_{\text{LED}}}{E_{\text{photon}}}
	      \end{align*}
	      Ta viết lại (5.58)
	      \begin{align*}
		      \langle i \rangle_{\text{photon}}
		       & = N_{\text{photonelectron}} \times e                                  \\
		       & = \eta_{\text{Detec}} e \Phi                                          \\
		       & = \eta_{\text{Detec}} e \f{i_{1} V_{\text{nguồn}}}{E_{\text{photon}}} \\
		       & = \eta_{\text{Detec}} e \f{i_{1}}{e} \eta_{\text{LED}} \eta_{2} \\
		       & = 2.145 \text{mA}
	      \end{align*}
	\item[(d)] Tổng hiệu suất lượng tử
	      \begin{align*}
		      \eta_{\text{tolta}} = \eta_{\text{Detec}} \times \eta_{\text{LED}} \times \eta_{\text{optic}} = 0.288  \tag{*}
	      \end{align*}
	      (*) Bachor, HA., Rottengatter, P. $\&$ Savage, C.M. Correlation effects in light sources with high quantum efficiency. Appl. Phys. B 55, 258–264 (1992).\\ https://doi.org/10.1007/BF00325014 \\
	      Hệ số $F_{\text{Fano}}$ được cho bởi
	      \begin{align*}
		      F_{\text{Fano}} = \eta_{\text{tolta}} F_{\text{drive}} + (1 - \eta_{\text{tolta}})
		       & = 0.288 \times 6.77 \times 10^{-3} + (1 - 0.288) \\
		       & = 0.7139
	      \end{align*}
	\item[(e)]
	      \begin{align*}
		      P_{\text{shot noise}}
		       & = 2 e R_{L} \Delta f \langle i \rangle_{\text{photon}}            \\
		       & = 2e \times 50 \times 10 \times 10^{3} \times 2.15 \times 10^{-3} \\
		       & = 3.44 \times 10^{-16} \text{W} \tag{1}
	      \end{align*}
	      \begin{align*}
		      F_{\text{photon}} = \f{P_\text{measured}}{P_{\text{photon}} \equiv P_{\text{shot noise}} } = 0.714 \Rightarrow P_\text{measured}
		       & = 0.714 \times 3.44 \times 10^{-16} \\
		       & = 2.45 \times 10^{-16} \text{W}
	      \end{align*}
\end{enumerate}






\end{document}
