\documentclass{article}
\usepackage[utf8]{vietnam}
\usepackage[utf8]{inputenc}
\usepackage{anyfontsize,fontsize}
\changefontsize[13pt]{13pt}
\usepackage{commath}
\usepackage{parskip}
\usepackage{xcolor}
\usepackage{amssymb}
\usepackage{slashed,cancel}
\usepackage{indentfirst}
\usepackage{pdfpages}
\usepackage{graphicx}
\usepackage{nccmath}
\usepackage{mathtools}
\usepackage{amsfonts}
\usepackage{amsmath,systeme,bbold,enumitem}
\usepackage[thinc]{esdiff}
\usepackage{hyperref}
\usepackage{bm,physics}
\usepackage{fancyhdr}
%footnote
\pagestyle{fancy}
\renewcommand{\headrulewidth}{0pt}%
\fancyhf{}%
\fancyfoot[L]{Vật lý Lý thuyết}%
\fancyfoot[C]{\hspace{4cm} \thepage}%

\usepackage{tikz-feynman}


\usepackage{geometry}
\geometry{
	a4paper,
	total={170mm,257mm},
	left=20mm,
	top=20mm,
}


\newcommand{\image}[2]{
	\begin{figure}[h!]
		\centering
		\includegraphics[width=0.5\textwidth]{pic/#1}
		\caption{#2}
	\end{figure}
}

\renewcommand{\l}{\ell}
\newcommand{\dps}{\displaystyle}

\newcommand{\f}[2]{\dfrac{#1}{#2}}
\newcommand{\at}[2]{\bigg\rvert_{#1}^{#2} }


\renewcommand{\baselinestretch}{1.5}


\title{\Huge{BTVN 1}}

\hypersetup{
	colorlinks=true,
	linkcolor=red,
	filecolor=magenta,      
	urlcolor=cyan,
	pdftitle={QO},
	pdfpagemode=FullScreen,
}

\urlstyle{same}

\begin{document}
\setlength{\parindent}{20pt}
\newpage
\author{TRẦN KHÔI NGUYÊN \\ VẬT LÝ LÝ THUYẾT}
\maketitle

\section{Quy tắc lọc lựa}
Ma trận lưỡng cực điện được cho bởi phương trình:
\begin{align}
	M_{12} = - \boldsymbol{\mu}_{12} \cdot \boldsymbol{{ \mathcal{E} }}_0,
\end{align}
với
\begin{align}
	\boldsymbol{\mu}_{12} = -e \left( \bra{2}x\ket{1} \hat{\textbf{\textsc{i}}} + \bra{2}y\ket{1} \hat{\textbf{\textsc{j}}} + \bra{2}z\ket{1} \hat{\textbf{\textsc{k}}} \right)
\end{align}
là moment lưỡng cực điện của sự chuyển dời từ trạng thái $1\rightarrow 2$. Khi có quá nhiều trạng thái cần phải tính toán, quy tắc lọc lựa sẽ cho phép chúng ta kết luận rằng một số thành phần phần tử ma trận nhất định là bằng \textit{không} mà không cần phải tính toán một cách chính xác.

Quy tắc lọc lựa lưỡng cực điện liên quan tới các bộ số lượng tử $l,m,s$ và $m_s$, được tổng hợp trong bảng dưới.

\begin{table}[h!]
	\centering
	\begin{tabular}{c | c}
		\hline
		Số lượng tử  & Quy tắc lọc lựa                      \\ [0.6ex]
		\hline
		Tính chẵn/lẻ & Thay đổi                             \\
		$\l$         & $\Delta \l = \l' - \l = \pm 1$       \\
		$m$          & $\Delta m = m' - m = 0$ hoặc $\pm 1$ \\
		$s$          & $\Delta s = s' - s = 0 $             \\
		$m_s$        & $\Delta m_s = m_s' - m_s = 0$        \\
		\hline
	\end{tabular}
	\caption{Quy tắc lọc lựa lưỡng cực điện cho nguyên tử một electron(hydrogen-like atom).}
\end{table}
\newpage
\noindent Quy tắc:
\image{selectionrule.png}{Phân rã cho phép cho bốn mức Bohr trong nguyên tử Hydro[Griffiths, David J., and Darrell F. Schroeter. 2018. Introduction to Quantum Mechanics. 3rd ed.]}

Sự chuyển dời tuân theo quy tắc lọc lựa lưỡng cực điện được gọi là \textbf{chuyển dời cho phép}, trong khi những chuyển dời không tuân theo quy tắc thì đươc gọi là \textbf{chuyển dời cấm}.

\section{Dáng điệu phổ}

Xét một ``vạch phổ'', như thường được quan sát trong các môn thực nghiệm. Những tưởng ``vạch'' ấy chỉ đơn giản là một vạch theo nghĩa toán học, tức chỉ một giá trị hoàn toàn xác định. Tuy nhiên, thực tế ``vạch'' ấy lại là một tập hợp nhiều giá trị xung quanh một giá trị nào đấy, tạo nên một ``phân bố'' có dạng hình chuông, gọi là hàm ``spectral lineshape'' $g_\omega(\omega)$.
Ta dễ thấy dạng hình ``quốc dân'' của phân bố Gauss:
\image{gauss.jpg}{Phân bố Gauss}

hoặc cũng có thể là Lorentz(Hình 3).

Giá trị trung tâm được xác định bởi 
\begin{align*}
	\hbar\omega_0 = E_2 - E_1.
\end{align*} 
và được chuẩn hóa:
\begin{align*}
	\int_{0}^{\infty}g_\omega(\omega)d\omega = 1.
\end{align*}

\image{cauchy.png}{Phân bố Lorentz}

Đại lượng quan trọng nhất cần được xác định là Full Width at Half Maximum (FWHM), tức độ rộng tại phân nửa độ cao cực đại, cho ta biết độ rộng của phổ.\\
Thường có ba nguyên nhân ảnh hưởng đến độ rộng phổ:
\begin{itemize}
	\item Thời gian sống (lifetime (natural) broadening)
	\item Va chạm (collisional (pressure) broadening)
	\item Doppler
\end{itemize}
\subsection{Lifetime broadening}
Ta đã biết liên hệ giữa hệ số Einstein A và thời gian sống $\tau$. Thời gian sống này liên hệ với năng lượng bức xạ (tức sự mở rộng của phổ) qua Bất định Heisenberg $$\Delta E\Delta t \geq \hbar$$
và $$\Delta E = \hbar\Delta\omega$$
nên $$\Delta \omega=\frac{\Delta E}{\hbar} \leq \frac{1}{\tau}$$
Ta có hàm mô tả dạng phổ $$g(\omega) = \frac{\Delta\omega}{2\pi} \frac{1}{(\omega - \omega_0)^2 + (\Delta\omega/2)^2}$$
và dáng hình của nó:
\image{g(w).png}{Phân bố Lorentz. Công thức được cho bởi $g(\omega)$.}
\subsection{Collisional broadening}
Các phân tử có thể va đập, với nhau và với thành bình, khiến thời gian ``sống'' bị giảm đi (năng lượng bị mất nhanh hơn).\\
Bằng thống kê, ta có: $$\tau_{\text{collision}} \sim \frac{1}{\sigma_s P} \left( \frac{\pi m k_B T}{8} \right)^{1/2}$$ với $\sigma_s$ là tiết diện va chạm và P là áp suất. Khi ấy, collisional broadening còn được gọi là pressure broadening.\\
Trong trường hợp này, thời gian sống thường là rất ngắn, dẫn đến độ rộng phổ thật lớn so với trường hợp ``natural''.\\
Vậy nên ta có thể khắc phục bằng cách giảm thiểu áp suất P, đó là lí do ta sử dụng các đèn áp suất thấp trong khảo sát phổ.\\
\image{pbroad.png}{Cơ chế Dobler broadening.}
\subsection{Doppler broadening}
Việc các hạt nguyên tử di chuyển lại gần hay ra xa khỏi máy đo cũng ảnh hưởng đến phổ, khi ta xét hiệu ứng Doppler là đáng kể. \\
Khi hạt di chuyển lại gần máy đo, ta có $$\omega=\omega_0\left(1+\frac{v_x}{c}\right)$$
Số hạt di chuyển với vận tốc $v_x$ đến $v_x+dv_x$ được cho bởi cơ học thống kê: $$N(v_x) = N_0 \left( \frac{2k_B T}{\pi m} \right)^{1/2} \exp \left( -\frac{m v_x^2}{2k_B T} \right)$$ với FWHM được cho bởi $$\Delta \omega_{\text{Doppler}} = 2 \omega_0 \left( \frac{(2 \ln 2) k_B T}{mc^2} \right)^{1/2} = \frac{4 \pi}{\lambda} \left( \frac{(2 \ln 2) k_B T}{m} \right)^{1/2}$$

\section{Laser}(Light Amplification by
 Stimulated Emission of Radiation) \\
Từ ``laser'' là từ viết tắt của cụm từ ``Light Amplification by Stimulated Emission of Radiation'' (khuếch đại ánh sáng bằng phát xạ kích thích). Laser hoạt động lần đầu tiên được chứng minh vào năm 1960, và từ đó, laser đã trở thành công cụ thiết yếu trong quang học phi tuyến và lượng tử. Trong phần này, chúng ta sẽ đưa ra một cái nhìn tổng quan ngắn gọn về các nguyên lý vật lý cơ bản của hoạt động laser, và sau đó mô tả ngắn gọn các đặc tính chính của các loại laser thường được sử dụng trong phòng thí nghiệm.
\subsection{Phát xạ cưỡng bức (Stimulated Emission)}
Nếu như một hạt đang ở trạng thái ``cao hơn'', tạm gọi là trạng thái \textit{upper}, được chiếu bởi một nguồn sáng, hạt đó có thể chuyển dời xuống vị trí ``thấp hơn'' - \textit{lower}, xác suất chuyển dời từ \textit{upper} $\rightarrow$ \textit{lower} là như nhau đối với xác suất chuyển dời từ \textit{lower} $\rightarrow$ \textit{upper}. Quá trình này được gọi là \textbf{Phát xạ cưỡng bức} hay \textbf{Phát xạ cảm ứng}.
\image{stimulatedemission.png}{Phát xạ cưỡng bức [Griffiths, David J., and Darrell F. Schroeter. 2018. Introduction to Quantum Mechanics. 3rd ed.]}

Trường điện từ \textit{nhận} thêm năng lượng $\hbar \omega_0$ từ nguyên tử; chúng ta nói rằng một photon đi và và \textit{hai} photon bị bức ra - photon đi vào cộng với một photon từ quá trình chuyển dời được gây ra bởi photon đi vào(Hình 2). Điều này làm tăng khả năng khuếch đại, nếu như có rất nhiều nguyên tử, tất cả trong số đó đều đang ở trạng thái \textit{upper} được cảm ứng bởi một photon bay tới, một chuỗi phản ứng sẽ xảy ra. Photon bay vào tạo ra hai photon, hai photon đó tạo ra một bốn photon, và cứ thế. Chúng ta sẽ có một lượng lớn photon bay ra, tất cả chúng đều có cùng tần số. Đây chính là nguyên lý đằng sau Laser.
\subsection{Dao động Laser}
\image{fig4.8.png}{Sơ đồ nguyên lý của máy dao động Laser [Fox, Mark, 2006 Quantum Optics: An Introduction]}

Laser bao gồm một \textbf{buồng cộng hưởng} và hai gương đầu cuối được gọi là \textbf{gương bán mạ} và \textbf{gương phản xạ toàn phần} có độ phản xạ lần lượt là $R_1$ và $R_2$. Ánh sáng bật lại giữa hai gương đầu-cuối và được khuếch đại mỗi khi được truyền qua buồng cộng hưởng.

Khuếch đại ánh sáng sảy ra khi có buồng cộng hưởng được đo bởi hệ số khuếch đại $\gamma(\omega)$ định nghĩa bởi:
\begin{align}
	\f{d I}{d z} = \gamma(\omega) I(z),
\end{align}
với	$I$ là cường độ quang, $\omega$ là tần số góc của ánh sáng, và $z$ là hướng chuyền của chùm tia. Lấy tích phân của phương trình (3) ta được:
\begin{align}
	I(z) = I_0 e^{\gamma z}.
\end{align}
Ý nghĩa vật lý: phương trình (4) cho thấy cường độ ánh sáng tăng theo hàm exponential bên trong buồng cộng hưởng.

Xét trường hợp chùm tia sáng là gần cộng hưởng với sự chuyển dời của nguyên tử với tần số góc là $\omega_0$. Chùm tia sẽ cảm ứng với cả hấp thụ và phát xạ cưỡng bức. Để xuất hiện khuếch đại, chúng ta ràng buộc rằng tốc độ phát xạ cưỡng bức phải lớn hơn tốc độ hấp thụ để cho photon trong chùm tia được tăng lên mỗi khi được truyền qua buồng cộng hưởng:
\begin{align}
	B_{21}^{\omega} N_2 u(\omega) & > B_{12}^{\omega} N_1 u(\omega) \\
	\Rightarrow N_2               & > \f{g_2}{g_1}N_1
\end{align}
Khi xảy ra cân bằng nhiệt, tỉ lệ giữa $N_2,N_1$ được cho bởi phương trình Boltzmann
\begin{align*}
	\f{N_1}{N_2} = e^{\frac{\hbar \omega}{k_B T}},
\end{align*}
điều này có nghĩa là sẽ không bao giờ thõa mãn được phương trình (6), và cường độ ánh sáng phân rã bằng cường độ ánh sáng truyền qua bởi vì tốc độ hấp thụ	lớn hơn tốc độ phát xạ cưỡng bức. Từ đó phương trình (6) chỉ đúng khi và chỉ khi điều kiện không cân bằng. Điều này có thể quan sát được bằng cách ``bơm'' năng lượng vào buồng cộng hưởng để kích thích số lượng lớn nguyên tử đạt được trnạg thái kích thích.

\subsection{Tính chất của Laser}
Laser được phân ra thành các loại dựa trên chất hóa học bên trong buồng cộng hưởng, đó là laser rắn, laser lỏng, laser khí. Hai đặc điểm chung cho tất cả các loại laser là tính định hướng của chùm tia và độ đơn sắc cao. Ngoài ra còn có cường độ lớn, có tính hợp cao.

\section{Bài tập}
\subsection*{Problem 4.2}
Toán tử chẵn/lẻ:
\begin{align*}
	\hat{\Pi} \psi(x) = \psi(-x).
\end{align*}
Xét ba chiều:
\begin{align*}
	\hat{\Pi} \psi(\mathbf{r}) = \psi(-\mathbf{r}).
\end{align*}
Ta biết toán tử moment lưỡng cực $\hat{\textbf{p}}_e$ tỉ lệ với $\mathbf{r}$. Khi tác động toán tử chẵn/lẻ lên $\hat{\textbf{p}}_e$ ta được:
\begin{align*}
	\hat{\Pi}^{\dagger} \hat{\textbf{p}}_e \hat{\Pi} = - \hat{\textbf{p}}_e.
\end{align*}
Ta viết lại trạng thái $\ket{2},\ket{1}$ dưới dạng bộ số lượng tử và có được:
\begin{align*}
	\boldsymbol{\mu}_{12}
	 & = \bra{n'\l'm'} \hat{\textbf{p}}_e \ket{n\l m}                                   \\
	 & =  - \bra{n'\l'm'}  \hat{\Pi}^{\dagger} \hat{\textbf{p}}_e \hat{\Pi} \ket{n\l m} \\
	 & =  - \bra{n'\l'm'}  (-1)^{\l'} \hat{\textbf{p}}_e (-1)^{\l} \ket{n\l m}          \\
	 & =  (-1)^{\l' + \l + 1} \bra{n'\l'm'}   \hat{\textbf{p}}_e \ket{n\l m},           
\end{align*}
dấu ``$=$'' xảy ra khi và chỉ khi $ (-1)^{\l' + \l + 1} = 1 \Rightarrow \l + \l' = odd \Rightarrow \l \neq \l'$ DPCM .

\subsection*{Problem 4.3}
Hàm sóng cho nguyên tử Hydro được viết dưới dạng:
\begin{align}
	\psi (r,\theta,\varphi) = F(r,\theta) e^{i m_{\l} \varphi },
\end{align}
với $ m_{\l}$ là số lượng tử từ.\\
\begin{table}[h!]
	\centering
	\begin{tabular}{|l  l|}
		\hline
		$Y_{0}^{0} = \left(\frac{1}{4\pi}\right)^{1/2}$                                  & $Y_{2}^{\pm 2} = \left(\frac{15}{32\pi}\right)^{1/2} \sin^2\theta e^{\pm 2i\varphi}$                     \\
		$Y_{1}^{0} = \left(\frac{3}{4\pi}\right)^{1/2}\cos\theta$                        & $Y_{2}^{0} = \left(\frac{7}{16\pi}\right)^{1/2} (5\cos^3\theta - 3 \cos\theta)$                          \\
		$Y_{1}^{\pm1} = \mp\left(\frac{3}{8\pi}\right)^{1/2}\sin\theta e^{\pm i\varphi}$ & $Y_{3}^{\pm 2} = \mp\left(\frac{21}{64\pi}\right)^{1/2} \sin\theta (5\cos^2\theta - 1)e^{\pm 2i\varphi}$ \\
		$Y_{2}^{0} = \left(\frac{5}{16\pi}\right)^{1/2}(3\cos^2 \theta - 1)$             & $Y_{3}^{\pm3} = \left(\frac{105}{32\pi}\right)^{1/2} \sin^2\theta\cos\theta e^{\pm 3i\varphi}$           \\
		\hline
	\end{tabular}
	\caption{Một số hàm điều hòa cầu $Y_{l}^{m}(\theta,\varphi)$.}
\end{table}\\
Từ Bảng 2, ta xây dựng công thức tổng quát cho (7):
\begin{align}
	\psi_{l}^{m_{\l}} (r,\theta,\varphi) = F(r,\theta) e^{i m_{\l} \varphi }.
\end{align}
Sử dụng toán tử quay $\hat{R}_n(\phi)(n = x, y ,z)$ tác động lên (8), ta có trị riêng của $\hat{R}_z(\phi)$:
\begin{align*}
	\hat{R}_z(\phi) \psi_{l}^{m_{\l}} (r,\theta,\varphi) = \psi_{l}^{m_{\l}}(r,\theta,\varphi') = e^{-2i\pi m_\l \phi}\psi_{l}^{m_{\l}}(r,\theta,\varphi),
\end{align*}
và
\begin{align*}
	 & \hat{R}^{\dagger}_z(\phi)\hat{R}_z(\phi) = 1,                                  \\
	 & \hat{R}^{\dagger}_z(\phi) z \hat{R}_z(\phi) = z,                               \\
	 & \hat{R}^{\dagger}_z(\phi) (x\pm iy) \hat{R}_z(\phi) = e^{\mp 2i\pi} (x\pm iy).
\end{align*}
Sử dụng kí hiệu Dirac cho hàm sóng (8) và để cho phép tính được gọn gàng ta đặt $\ket{\psi_{l}^{m_{\l}} (r,\theta,\varphi)} = \ket{m_\l}$, tính các thành phần ma trận sau:
\begin{enumerate}[label=(\alph*), left=0pt, itemindent=*, align=left]
	\item
	      \begin{align*}
		      \bra{m'_\l} \hat{z} \ket{m_\l}
		       & = \bra{m'_\l} \hat{R}^{\dagger}_z(\phi)\hat{R}_z(\phi) \hat{z} \hat{R}^{\dagger}_z(\phi)\hat{R}_z(\phi)\ket{m_\l} \\
		       & = e^{2i\pi m'_\l \phi} e^{-2i\pi m_\l \phi} \bra{m'_\l} \hat{z} \ket{m_\l}.
	      \end{align*}
	      Để dấu ``='' xảy ra thì $m'_\l - m_\l = 0$  (ĐPCM).
	\item[(b),(c),(d)]
		\begin{align*}
			\bra{m'_\l} \hat{z} \ket{m_\l}
			 & = \bra{m'_\l} \hat{R}^{\dagger}_z(\phi)\hat{R}_z(\phi) (\hat{x} \pm i\hat{y}) \hat{R}^{\dagger}_z(\phi)\hat{R}_z(\phi)\ket{m_\l} \\
			 & = e^{2i\pi m'_\l \phi} e^{-2i\pi m_\l \phi} e^{\mp 2i\pi} \bra{m'_\l} \hat{x} \pm i\hat{y} \ket{m_\l}.
		\end{align*}
		Để dấu ``='' xảy ra thì,  $m'_\l - m_\l \mp 1= 0$ (ĐPCM).
\end{enumerate}
\subsection*{Problem 4.4}
\begin{enumerate}[label=(\alph*)]
	\item Ta có cường độ ánh sáng tỉ lệ với bình phương biên độ của điện trường
	\begin{align}
		\langle I \rangle \propto  \langle \mathcal{E}(t)^2 \rangle.
	\end{align}
	Từ 
	\begin{align}
		I(t) = I(0) \exp(-t/\tau),
	\end{align}
	chúng ta có thể suy ra rằng
	\begin{align}
		\abs{\mathcal{E}(t)}^2 = \abs{\mathcal{E}_0}^2 \exp(-t/\tau).
	\end{align}
	Bên cạnh đó, ta có thể khai triển phương trình (11) dưới dạng hàm $\cos$
	\begin{align}
		\mathcal{E}(t) = \mathcal{E}_0 \cos(\omega_0 t) \exp(-t/2\tau)
	\end{align}
	mà vẫn thỏa mãn điều kiện rằng cường độ $I(t)$ tỉ lệ với bình phương biên độ điện trường.
	\begin{itemize}
		\item Với $t<0, \mathcal{E}(t) = 0$, bởi vì chùm tia sáng vẫn chưa được bức xạ ra.
		\item Với $t \geq 0$, trường điện có công thức là (12), mô tả tính chất dao động của sóng ánh sáng với biên độ giảm dần theo hàm $\exp$, phù hợp với cường độ ánh sáng đã cho. Để tìm phổ phát xạ, chúng ta thực hiện phép biến đổi Fourier của điện trường.
	\end{itemize}
	\item Trường điện với $t \geq 0$ được cho bởi công thức
	\begin{align*}
		\mathcal{E}(t) 
		& = \mathcal{E}_0 \cos(\omega_0 t) \exp(-t/2\tau) \\
		& = \f{\mathcal{E}_0}{2} \cos(\omega_0 t) \exp(-t/2\tau)
	\end{align*}
	Phép biển đổi Fourier cho $\mathcal{E}(t)$ là
	\begin{align}
		\mathcal{E}(\omega) = \f{1}{\sqrt{2}} \int_{-\infty}^{+\infty} \mathcal{E}(t) e^{i\omega t} dt.
	\end{align} 
	Khi $\mathcal{E}(t) = 0$, với $t < 0$, phương trình (13) trở thành
	\begin{align}
		\mathcal{E}(\omega)
		&= \f{1}{\sqrt{2}} \int_{0}^{+\infty} \left( \f{\mathcal{E}_0}{2} e^{i(\omega_0 - \omega)t - t/2\tau} + \f{\mathcal{E}_0}{2} e^{-i(\omega_0 + \omega)t - t/2\tau} \right) dt \nonumber \\
		& = \frac{\mathcal{E}_0}{2\sqrt{2\pi}} \left( \frac{1}{1/2\tau - i(\omega_0 - \omega)} + \frac{1}{1/2\tau + i(\omega_0 + \omega)} \right).
	\end{align}
	Giả sử rằng $\omega_0 \gg 1/\tau$,
	\begin{align*}
		\mathcal{E}(\omega) \approx \frac{\mathcal{E}_0}{2\sqrt{2\pi}} \cdot \frac{1}{1/2\tau - i(\omega_0 - \omega)}.
	\end{align*}(ĐPCM)
	\item  Cường độ ánh sáng tỉ lệ với bình phương biên độ, nên ta có
	\begin{align}
		I(\omega) \propto |\mathcal{E}(\omega)|^2 = \left|\frac{\mathcal{E}_0}{2\sqrt{2\pi}} \cdot \frac{1}{1/2\tau - i(\omega_0 - \omega)}\right|^2.
	\end{align}
	Ta có thể nói rằng
	\begin{align}
		I(\omega) \propto \f{1}{(\omega - \omega_0)^2 + (1/2\tau)^2},
	\end{align}
	điều này hoàn toàn đúng so với công thức được cho.
\end{enumerate}
\subsection*{Problem 4.6}
Ta biết khối lượng nguyên tử của $M_{Na} = 23$ và bán kính nguyên tử $r_{Na} \approxeq 0.2nm = 0.2\times 10^{-10}$, tiết diện tán xạ đàn hồi $\sigma_s$ là
\begin{align}
	\sigma_s = \pi (r_{Na})^2 = 4\pi \times 10^{-20} (m^2)
\end{align}  
Ở ĐKTC, $T = 273 K$ và $P = 100k Pa$. Thời gian tán xạ đàn hồi là
\begin{align*}
	\tau_{\text{collision}} 
	&= \f{1}{\sigma_s P} \left( \f{\pi m k_{\text{B} T}}{8} \right)^{1/2} \\
	& = \f{1}{4\pi \times 10^{-20 } \times 100\times10^{3}} \left( \f{\pi \frac{23}{N_A} \times 10^{-3} \times 1.38\times 10^{-23} \times 273 }{8} \right)^{1/2}\\
	& \approx 5.9 \times 10^{-10} (s)
\end{align*}






















































































\end{document}